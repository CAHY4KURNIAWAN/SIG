\documentclass{WileySix}

\usepackage{w-bookps}
\usepackage{graphicx}
\usepackage{listings}
\lstset{
   breaklines=true,
   basicstyle=\ttfamily}
\usepackage{amsmath}

\setcounter{secnumdepth}{3}
\setcounter{tocdepth}{2}

\begin{document}


\booktitle{Pengantar\\ Sistem Informasi\\ Geografis}
\subtitle{Sejarah, Definisi dan Konsep Dasar}

\author{Rolly Maulana Awangga}

\halftitlepage
\titlepage

\offprintinfo{Pengantar Sistem Informasi Geografis}{Rolly Maulana Awangga}


\begin{copyrightpage}{Juni 2018}
Pengantar Sistem Informasi Geografis
\end{copyrightpage}


\dedication{For my family Bunda, Khafayah dan Khanza}

%\dedication{To my parents, Mamah Umiyatin dan Papah Waryo}

\begin{preface}
	Saya ucapkan alhamdulillah atas selesainya buku ini. Berkat semua dukungan rekan dan mahasiswa alhamdulillah buku ini bisa terselesaikan. Terima kasih untuk keluarga kecil saya bunda Mila, Khafa dan Khanza yang senantiasa menemani dan mendukung seluruh aktifitas penulis. Tidak lupa pula terima kasih kepada rekan-rekan mahasiswa bimbingan yang membantu layouting dan editing agar buku ini bisa terbit.

	Buku ini merupakan buku hasil pemikiran dan uji coba serta praktek lapangan selama mendalami Sistem Informasi Geografis. Diawali dengan kebutuhan akan aplikasi open source pada Kementrian Sekretariat Negara yang menolak menggunakan Google Map untuk dijadikan base map di Aplikasinya. Hal ini dikarenakan untuk mengendalikan kedaulatan data rahasia milik negara yang tidak ingin dikirimkan ke pihak manapun apalagi pihak luar. Yang akhirnya saya mencoba berbagai macam alternatif untuk membangun sendiri server peta dan kelengkapan penggunaannya. Seiring perkembangan waktu dan kerja sama dengan berbagai pihak. Ternyata memang banyak kebutuhan GIS dengan aplikasi open source ini dengan alasan yang sama yaitu kedaulatan data. Selain itu open source yang membawa jargon “freedom” sesuai dengan filosofi keamanan data tanpa ketergantungan dengan berbagai aplikasi Propietary. Dengan aplikasi open source juga bisa menekan biaya pengeluaran instansi dalam hal pengeluaran pembelian piranti lunak GIS. Termasuk dalam hal pembelajaran di kampus. Kampus tidak perlu lagi membeli piranti lunak mahal, dan mahasiswa tidak perlu untuk berlangganan kepada Google Maps yang cukup menguras uang ketika akan mengimplementasikannya di tempat kerja. Mungkin kita sudah mengetahui betapa mudahnya menggunakan Google Maps API pada saat kuliah. Akan tetapi ketika diterapkan pada sebuah proyek GIS, nilainya akan tinggi sekali untuk pembayaran penggunaan API tersebut. Belum lagi masalah keamanan data yang terus di pantau oleh Google.

	Semoga kedepannya buku ini akan terus dikembangkan dan akan semakin dilengkapi dengan pengalaman-pengalaman baru penulis di edisi selanjutnya. Go Open Source Go Freedom.

\where{Bandung, Juni 2018}
Penulis	
\prefaceauthor{Rolly Maulana Awangga}
\end{preface}

%\contentsinbrief %optional
\tableofcontents
\listoffigures %optional
\listoftables  %optional


%\begin{foreword}
%Terima Kasih Atas semua dukungannya
%\end{foreword}




%\acknowledgments
%acknowledgment buat semuaya
%\authorinitials{RMA} % ie, I. R. S.


%%%%%%%%%%%%%%%%%%%%%%%%%%%%%%%%
%% Glossary Type of Environment:

% \begin{glossary}
% \term{<term>}{<description>}
% \end{glossary}

%%%%%%%%%%%%%%%%%%%%%%%%%%%%%%%%
% \begin{acronyms} 
% \acro{<term>}{<description>}
% \end{acronyms}

%%%%%%%%%%%%%%%%%%%%%%%%%%%%%%%%
%% In symbols environment <term> is expected to be in math mode; 
%% if not in math mode, use \term{\hbox{<term>}}

% \begin{symbols}
% \term{<math term>}{<description>}
% \term{\hbox{<non math term>}}Box used when not using a math symbol.
% \end{symbols}

%%%%%%%%%%%%%%%%%%%%%%%%%%%%%%%%
%\begin{introduction}
%\introauthor{Rolly Maulana Awangga}{Program Studi Sarjana Terapan Teknik Informatika, Politeknik Pos Indonesia}
% Perkenalkan saya Rolly Maulana Awangga Dosen Politeknik Pos Indonesia dengan mata kuliah
%Sistem Informasi Geografis, Kecerdasan Buatan dan Sistem Multimedia
%\end{introduction}

\part[Pendahuluan]
{Definisi, Teori\\ dan Sejarah Geospasial}

\chapter[Definisi]
{Pendahuluan\\ Definisi}
\input{chapter/definisi.tex}

\chapter[Sejarah Ptolemy]
{Pendahuluan\\ Sejarah Ptolemy}
\input{chapter/ptolemy.tex}

\chapter[Sejarah eratosthenes]
{Pendahuluan\\ Sejarah eratosthenes}
\input{chapter/eratosthenes.tex}


\chapter[Sejarah aliddrissi]
{Pendahuluan\\ aliddrissi}
\input{chapter/aliddrissi.tex}

\chapter[Sejarah Peta Dinding]
{Pendahuluan\\ Sejarah Peta Dinding}
\input{chapter/willemblaeu.tex}

\chapter[Sejarah Bumi]
{Pengantar\\ Sejarah Bumi}
% Nama Kelompok : Kelompok 4
% Akbar Pambudi Utomo (1154094) 
% Andi Nurfadilah Ali (1154041)
% Andi Wadi Afryandika (1154113)
% Hanna Theresia Siregar (1154009)
% Julham Ramadhana (1154069)
% Pebridayanti Hasibuan (1154118)

\section{Sejarah Bumi}
Bumi merupakan planet atau rumah kita dalam kedudukan di tata surya. Peradaban kuno percaya bahwa bumi itu datar, dengan langit berputar-putar sekali sehari. Secara umum yang diyakini bahwa kehidupan di Bumi dimulai di Bumi itu sendiri, beberapa waktu setelah terbentuknya planet antara 4000-5000 juta tahun yang lalu. Namun ada yang berpendapat bahwa kehidupan di luar bumi itu ada, tetap kita tidak memiliki bukti pasti tentang kehidupan di tempat lain. Yang perlu kita ketahui bumi berada pada galaksi bimasakti dimana terdapat matahari sebagai sistem bintang.

Dalam Geologi sendiri atau biasa disebut sebagai ilmu pengetahuan tentang Kebumian yang mempelajari segala sesuatu mengenai planet Bumi beserta isinya yang pernah ada. Dalam Geologi juga akan dibahas tentang sifat-sifat dan bahan- bahan yang membentuk bumi itu apa, serta struktur dan proses-proses yang bekerja baik di dalam maupun dibagian teratas permukaan bumi, kedudukannya di Alam Semesta hingga sekarang. Geologi merupakan ilmu pengetahuan yang kompleks, mempunyai pembahasan materi yang beraneka ragam namun juga merupakan ilmu pengetahuan yang bagus untuk dipelajari. Sebagai landasan prinsip untuk dapat mempelajari ilmu geologi adalah bahwasanya kita harus menganggap bumi ini sebagai suatu benda yang secara dinamis berubah sepanjang masa, setiap saat dan setiap detik. 
Pemikiran geologi modern dikenal kan oleh Huttonian revolution mengemukakan pemikiran-pemikirannya sebagai berikut:
\begin{enumerate}
\item Bahwasanya proses-proses alam yang sekarang terjadi, menyebabkan perubahan pada permukaan bumi, juga bekerja sepanjang umur dari bumi ini. 
\item Ia juga mengamati bahwa proses-proses tersebut yang walaupun bekerja sangat lambat, tetapi pada akhirnya mampu menyebabkan terjadi nya perubahan-perubahan yang sangat besar pada bumi. 
\item Bahwa bumi ini sangat dinamis, yang berarti mengalami perubahan-perubahan secara terus-menerus mengikuti suatu pola daur (siklus) yang berulang-ulang.
\end{enumerate}
Bumi sendiri berada di kawasan di mana terjadi nya tumpang tindih antara litosfer (daratan) bagian padat dari bumi, hidrosfer (perairan), dan atmosfer (udara) yang menyelubungi bumi dengan zarah-zarah dan benda-benda yang mengisinya.
Dalam sejarah terbentuknya bumi sewaktu SMA kita pernah mempelajari teori-teori terbentuknya bumi dalam pelajaran geografi \cite{wetherill1990formation}.

\subsection{Teori-teori terbentuknya Bumi}
\subsubsection{Teori Kabut Kant-Laplace}
\begin{figure} [ht]
	\centerline{\includegraphics[width=.5\textwidth]{figures/teorikabutnebula.JPG}}
	\caption{Gambar Teori Nebula}
	\label{teorikabutnebula}
	\end{figure}

Gambar \ref{teorikabutnebula} adalah gambar dari Teori Kabut Nebula. Teori ini dikenal dengan teori kabut (nebula) yang dikemukakan oleh Immanuel Kant (1755) dan Pierre de Laplace (1796). dalam teori ini dikemukakan bahwa di jagat raya terdapat gas yang kemudian berkumpul menjadi kabut(nebula). gaya tarik-menarik antar gas ini membentuk kumpulan kabut yang sangat besar dan berpusat semakin cepat sehingga materi kabut bagian khatulistiwa ter lempar memisah dan memadat(karena pendinginan), bagian yang ter lempar inilah yang kemudian menjadi sebuah planet dalam tata surya. Bumi baru terus bertumbuh sampai suhu interior nya cukup panas untuk melelehkan logam siderofil. Dengan massa jenis yang lebih tinggi dari silikat, membuat logam ini menjadi tenggelam. Proses ini terjadi 10 juta tahun setelah Bumi mulai terbentuk, dan menghasilkan struktur Bumi yang berlapis-lapis dan mengakibatkan terbentuknya medan magnet. J. A. Jacobs merupakan orang pertama yang menunjukkan bahwa inti dalam bagian dalam yang padat berbeda dari inti luar yang padat—membeku dan mengembang keluar inti luar yang cair dikarbagus untuk bagian dalam bumi yang makin mendingin (sekitar 100° C per miliar tahun. Ekstrapolasi dari pengamatan ini memperkirakan bahwa inti terbentuk pada masa 2–4 miliar tahun yang lalu. Jika ini benar maka berarti bahwa inti bumi bukanlah fitur primordial yang berasal selama pembentukan planet.

\subsubsection{Teori Planetesimal}

\begin{figure} [ht]
	\centerline{\includegraphics[width=1\textwidth]{figures/teoriplanetesimal.JPG}}
	\caption{Gambar Teori Planetisimal}
	\label{teoriplanetesimal}
	\end{figure}

	Pada Gambar berikut \ref{teoriplanetesimal} adalah gambar dari Teori Planetisimal.
seabad kemudian sesudah teori kabut tersebut muncul teori Planetisimal yang dikemukakan oleh Chamberlin dan Moulton. Teori ini mengungkapkan bahwa pada mulanya telah terdapat Matahari asal. Pada suatu ketika, matahari asal ini di dekati sebuah bintang besar yang menyebabkan terjadi nya penarikan pada bagian matahari. Akibat tenaga tarik menarik tadi, terjadilah ledakan yang dahsyat. Gas yang meledak ini keluar dari atmosfer matahari, kemudian mengembun dan membeku sebagai benda-benda yang padat(disebut planetesimal). Planetisimal ini dalam perkembangannya menjadi planet-planet, dan salah satunya planet bumi kita.


Pada Gambar \ref{teoriplanetesimal} adalah gambar dari Teori Planetesimal. Seabad kemudian sesudah teori kabut tersebut muncul teori Planetesimal yang dikemukakan oleh Chamberlin dan Moulton. Teori ini mengungkapkan bahwa pada mulanya telah terdapat Matahari asal. pada suatu ketika, matahari asal ini didekati sebuah bintang besar yang menyebabkan terjadinya penarikan pada bagian matahari. Akibat tenaga tarik menarik tadi, terjadilah ledakan yang dahsyat. Gas yang meledak ini keluar dari atmosfer matahari, kemudian mengembun dan membeku sebagai benda-benda yang padat(disebut planetesimal). Planetesimal ini dalam perkembangannya menjadi planet-planet, dan salah satunya planet bumi kita.

\subsubsection{Teori Pasang Surut Gas}
\begin{figure} [ht]
	\centerline{\includegraphics[width=1\textwidth]{figures/teoripasangsurut.JPG}}
	\caption{Gambar Pasang Surut}
	\label{teoripasangsurut}
	\end{figure}

	Pada Gambar berikut \ref{teoripasangsurut} adalah gambar dari Teori Pasang Surut.
Teori ini dikemukakan oleh Jeans dan Jeffreys, yakni bahwa sebuah bintang besar mendekati matahari dalam jarak pendek, sehingga menyebabkan terjadi nya pasang surut pada tubuh matahari. Dalam lidah yang panas ini terjadi perapatan gas-gas dan akhirnya kolom-kolom ini akan pecah, lalu berpisah menjadi benda-benda tersendiri yaitu planet-planet. Bintang besar yang menyebabkan penarikan pada bagian-bagian tubuh matahari tadi melanjutkan perjalanan di jagat raya, sehingga lambat laun akan hilang pengaruhnya terhadap planet-planet yang terbentuk tadi, lalu planet-planet itu akan mengelilingi matahari dan mengalami proses pendinginan, proses pendinginan berjalan lambat pada planet besar seperti Yupiter dan Saturnus, sedangkan planet kecil seperti bumi mengalami proses pendinginan yang relatif lebih cepat.


Pada Gambar berikut \ref{teoripasangsurut} adalah gambar dari Teori Pasang Surut.
Teori ini dikemukakan oleh Jeans dan Jeffreys, yakni bahwa sebuah bintang besar mendekati matahari dalam jarak pendek, sehingga menyebabkan terjadi nya pasang surut pada tubuh matahari. Dalam lidah yang panas ini terjadi perapatan gas-gas dan akhirnya kolom-kolom ini akan pecah, lalu berpisah menjadi benda-benda tersendiri yaitu planet-planet. Bintang besar yang menyebabkan penarikan pada bagian-bagian tubuh matahari tadi melanjutkan perjalanan di jagat raya, sehingga lambat laun akan hilang pengaruhnya terhadapt planet-planet yang terbentuk tadi, lalu planet-planet itu akan mengelilingi matahari dan mengalami proses pendinginan, proses pendinginan berjalan lambat pada planet besar seperti yupiter dan saturnus, sedangkan planet kecil seperti bumi mengalami proses pendinginan yang relatif lebih cepat.

\subsubsection{Teori Bintang Kembar}
\begin{figure} [ht]
	\centerline{\includegraphics[width=1\textwidth]{figures/teoribintangkembar.JPG}}
	\caption{Gambar Teori Bintang Kembar}
	\label{teoribintangkembar}
	\end{figure}


Pada Gambar \ref{teoribintangkembar} adalah gambar dari Teori Bintang Kembar.
Teori ini dikemukakan oleh seorang ahli astronomi R. A. Lyttleton. Menurut teori ini, galaksi berasal dari kombinasi bintang kembar. Salah satu bintang meledak sehingga banyak materi yang ter lempar. Karena bintang yang tidak meledak mempunyai gaya gravitasi yang masih kuat, maka sebaran pecahan ledakan bintang tersebut mengelilingi bintang yang tidak meledak. Bintang yang tidak meledak itu adalah matahari, sedangkan pecahan bintang yang lain itu adalah planet-planet yang mengelilinginya.

\subsubsection{Teori Dentuman Besar (Big Bang Teori)}

\begin{figure} [ht]
	\centerline{\includegraphics[width=1\textwidth]{figures/teoribigbang.JPG}}
	\caption{teori big bang}
	\label{teoribigbang}
	\end{figure}

	
Pada Gambar \ref{teoribigbang} adalah gambar dari Teori Dentuman Besar(Big Bang).
Pada Teori ini berdasarkan dari asumsi adanya massa yang sangat besar dan mempunyai massa jenis sangat besar. Adanya reaksi inti menyebabkan massa tersebut meledak hebat. Massa tersebut kemudian mengembang dengan sifat sangat cepat menjauhi pusat ledakan, karena adanya gravitasi, maka bintang yang paling kuat gravitasi nya akan menjadi pusatnya.
Dari berbagai teori, teori ini yang paling banyak didukung oleh para ilmuwan.

\section{Pendapat Tentang Sejarah Bumi}
Bumi terbentuk sekitar 4,54 miliar (4,54×109) tahun yang lalu melalui akresi dari nebula matahari. Pelepasan gas vulkanik diduga menciptakan atmosfer tua yang nyaris tidak beroksigen dan beracun bagi manusia dan sebagian besar makhluk hidup masa kini. Sebagian besar permukaan Bumi meleleh karena vulkanisme ekstrem dan sering bertabrakan dengan benda angkasa lain. Sebuah tabrakan besar diduga menyebabkan kemiringan sumbu Bumi dan menghasilkan Bulan. Seiring waktu, Bumi mendingin dan membentuk kerak padat dan memungkinkan cairan tercipta di permukaannya. Bentuk kehidupan pertama muncul antara 2,8 dan 2,5 miliar tahun yang lalu. Kehidupan fotosintesis muncul sekitar 2 miliar tahun yang lalu, nan memperkaya oksigen di atmosfer. Sebagian besar makhluk hidup masih berukuran kecil dan mikroskopis, sampai akhirnya makhluk hidup multiseluler kompleks mulai lahir sekitar 580 juta tahun yang lalu. Pada periode Kambrium, Bumi mengalami diversifikasi filum besar-besaran yang sangat cepat. Perubahan biologis dan geologis terus terjadi di planet ini sejak terbentuk. Organisme terus berevolusi, berubah menjadi bentuk baru atau punah seiring perubahan Bumi. Proses tektonik lempeng memainkan peran penting dalam pembentukan lautan dan benua di Bumi, termasuk kehidupan di dalamnya. Biosfer memiliki dampak besar terhadap atmosfer dan kondisi abiotik lainnya di planet ini, seperti pembentukan lapisan ozon, proliferasi oksigen, dan penciptaan tanah. Dalam sebuah artikel dari zuhdi2012sistem yang menyebutkan bahwa Bumi merupakan salah satu planet yang berada dalam tata surya yang diduga terbentuk dari pecahan-pecahan bintang pada jutaan tahun yang lalu, dan kemudian terperangkap pada gravitasi matahari sehingga akan selalu mengelilingi matahari. Menurut Hukum Newton kenapa planet dapat bertahan dalam pergerakan keliling atau biasa disebut revolusi dikarbagus untukan planet melakukan gerak melingkar yang menimbulkan gaya sentifugal yang besarnya  dengan gaya gravitasi namun berlawanan arah. Gaya gravitasi ini sendiri akan berkurang sesuai dengan semakin jauhnya jarak planet dari matahari, sedangkan gaya sentrifugal  akan tergantung pada kecepatan gerak melingkar planet. Semakin cepat gerakan tersebut maka akan semakin besar daya sentifugal. Bila secara kebetulan kedua gaya ini memiliki kecepatan yang sama besar, maka planet akan terjebak mengelilingi matahari. Pada saat tata surya terbentuk diperkirakan terdapat jutaan planet. Akan tetapi sebagian terjatuh ke matahari atau terlempar lepas dari pengaruh matahari. Selain berkeliling, planet juga akan bergerak memutari porosnya (rotasi). Gerak rotasi ini sendiri berlangsung dalam waktu lama sehingga membuat planet berbentuk seperti bola. Pada masa lalu, planet planet bukanlah sebuah benda padat, melainkan berupa magma atau berupa cairan batu. Bagian padat pada planet terbentuk selama proses pendinginan dan terjadi pada bagian kulit terluar dari planet tersebut. Bentuk Bumi sendiri yang dikatakan berbentuk bola tidaklah sempurna. Gerak Rotasi telah mengubah bentuk bumi menjadi agak cepat terhadap kedua kutub nya.\cite{zuhdi2012sistem}



Sejarah pembentukan Bumi yang dipelajari dalam materi pelajaran Geografi cenderung memiliki sifat abstrak yang akan lebih mudah dimengerti, jika memakai media yang cocok. Salah satu Inovasi pembelajaran yang tepat untuk dilakukan adalah menggunakan kartu indeks dan media film. Media seperti kartu indeks yang dipergunakan sebagai salah satu upaya yang memudahkan peserta didik agar mengingat konsep-konsep materi yang sedang dipelajari sedangkan media film sendiri merupakan media visual yang akan menjelaskan dengan lebih konkrit tentang fenomena bumi. Dalam sebuah artikel dari @article{widiyati2011meningkatkan menyebutkan bahwa Pembentukan Bumi dengan kategori Continental Drift Teori atau biasa disebut dengan teori pengepungan benua yang di temukan oleh Alfred Wegener pada tahun 1912 mengemukakan bahwa sampai sekitar 255 juta tahun lalu, di bumi baru ada satu benua dan samudra yang sangat luas. Benua raksasa ini sendiri dinamakan pangea, sedangkan kawasan samudra yang mengapit nya itu mengalami retakan-retakan dan pecah. Sekitar 135 juta tahun lalu, benua raksasa tersebut pecah menjadi dua, yaitu pecahan benua di sebelah utara yang dinamakan Lauransia dan di bagian selatan dinamakan gondwana.
\cite{widiyati2011meningkatkan}


%\chapter[Sejarah Kutub Utara]
%{Pendahuluan\\ Sejarah Kutub Utara}
%\input{chapter/kutubutara.tex}

%\chapter[Tentang Kutub Selatan]
%{Pendahuluan\\ Sejarah Antartika}
%\input{chapter/Antartika.tex}

\chapter[Sejarah Benua]
{Pendahuluan\\ Sejarah benua}
\input{chapter/benua.tex}

\chapter[Sejarah Penentuan Waktu]
{Pendahuluan\\ Sejarah Penentuan Waktu}
\input{chapter/sejarahwaktu.tex}

\chapter[Sejarah Penanggalan]
{Pendahuluan\\ Sejarah  Penanggalan}
%Define sejarah penanggalan, bulan dan tahun
%kelompok 4 D4 TI-2D
%Ayu Permata Sari        1154022
%Librantara Erlangga     1154071
%Martin Luter Zega       1154120
%Putri Aulia Ramadhanie  1154096
%Ryan Hafizh Herdiana    1154067
%Copyright (c) 2017 Copyright Holder All Rights Reserved.


\section{Sejarah penanggalan}

  Penanggalan merupakan salah satu sebuah mahakarya yang bisa ditemukan oleh umat manusia. Manusia mempelajari dan memanfaatkan alam [Matahari,Bulan dan Bintang] untuk menghitung pergantian tanggal, bulan dan juga tahun.
umumnya penanggalan digunakan untuk mengetahui waktu yang telah dilewati oleh umat manusia. Adanya sistem penanggalan ini membuat manusia dapat mengingat seluruh kejadian dan peristiwa yang terjadi di dunia ini.
Menurut artikel dari setyanto berdasarkan benda langit yang digunakan sebagai dasar perhitungan sistem penanggalan dapat dikategorikan menjadi 3 kelompok\cite{setyanto2015kriteria}.


  \subsection{Solar calendar/Kalender Surya}
    Kalender surya menggunakan pergerakan bumi mengelilingi matahari sebagai acuannya. Sistem kalender surya ini biasa digunakan oleh orang-orang Eropa. Beberapa contoh kalender yang menggunakan sistem ini yaitu:

    \subsubsection{Julian calendar/Kalender Julian}
      Kalender julian merupakan contoh kalender yang menerapkan sistem surya menurut artikel dari rachman planet\cite{rachmanplanet}.Kalender ini telah digunakan bahkan 45 tahun sebelum Masehi.
    Awalnya ketika Julius Caesar memimpin pemerintahan romawi terjadi kekacauan  pada perhitungan kalender yang menyebabkan Julis Caesar saat itu mengakhirinya dengan membuat perhitungan kalender sendiri dengan ketentuan:
    \begin{enumerate}
	  \item Satu tahun ditetapkan 565,25 Hari.
      \item Tahun biasa, yaitu tiga tahun berturut-turut yang harinya berjumlah 365 Hari.
      \item Tahun Kabisat, yaitu tahun keempat ditambah satu hari menjadi 366 Hari. Tambahannya dilakukan pada bulan Februari yang jika pada tahun biasa 28 hari pada tahun kabisat ini menjadi 29 hari.
      \item Titik permulaan musim semi/bunga ditetapkan pada tanggal 24 Maret.
      \item Permulaan tahun ditetapkan pada tanggal 1 Januari (Sebelumnya awal tahun ditetapkan pada tanggal 24 Maret).
	\end{enumerate}
    Meskipun kalender julian sudah sangat baik namun ternyata masih terdapat cacat pada kalender tersebut.
    Sebelum orang romawi menggunakan kalender julius caesar, orang romawi sudah menggunakan nama-nama bulan seperti:
    \begin{enumerate}
      \item  Martius     = 31 hari
      \item  Aprilis     = 29 hari
      \item  Majus       = 31 hari
      \item  Junius      = 29 hari
      \item  Quintilis   = 31 hari
      \item  Sextilis    = 29 hari
      \item  September   = 29 hari
      \item  October     = 31 hari
      \item  November    = 29 hari
      \item  Dcember     = 29 hari
      \item  Januarius   = 29 hari
      \item  Februarius  = 28 hari
    \end{enumerate} 

    \subsubsection{Gregorian calendar/Kalender Gregorius}
      Pada tahun 1582 Masehi Paus Gregorius menyaksikan musim semi/bunga pada tanggal 11 Maret, bukan lagi pada tanggal 24 maret seperti pada kalender julian. Kemudian paus gregorius memperbaikinya dengan cara:
\begin{enumerate}
      \item Musim semi/bunga ditetapkan pada tanggal 21 Maret.
      \item Tahun biasa menjadi 365 hari dan tahun kabisat menjadi 566 hari.
\end{enumerate}
      Kalender gregorius lebih dikenal dengan nama kalender Masehi yang jumlah hari pada setiap bulan dan penetapan awal tahun seperti yang digunakan kalender umumnya saat ini. Kalender Masehi dimulai dari tanggal 1 Januari
      tahun 1, pukul 00.00. Penamaan bulan pada kalender gregorius yang digunakan hingga sekarang:
      \begin{enumerate}
        \item  January   = 31 hari
        \item  February  = 28/29 hari
        \item  March     = 31 hari
        \item  April     = 30 hari
        \item  May       = 31 hari
        \item  June      = 30 hari
        \item  July      = 31 hari
        \item  August    = 31 hari
        \item  September = 30 hari
        \item  October  = 31 hari
        \item  November = 30 hari
        \item  December = 31 hari
      \end{enumerate}

 \subsection{lunar calendar/Kalender candra}

    \begin{figure}[ht]
    \centerline{\includegraphics[width=1\textwidth]{figures/Kalender_2015.JPG}}
    \caption{Kalender tahun 2015 Masehi / 1436 Hijriah.}
    \label{Kalender_2015}
    \end{figure}

    Pembahasan Kalender hijriah terkait dengan sistem penanggalan yang berpedoman pada pergerakan Bulan tampak dari Bumi yaitu ketika Matahari dan Bulan yang berada pada posisi bujur astronomi yang sama. Konjungsi merupakan pergerakan pada posisi Bulan dan Matahari yang telah disepakati sebagai batas penentuan secara astronomis pada Kalender Hijriah.

  Bulan yang berkonjungsi searah dengan Matahari akan tampak gelap pada permukaannya ketika dilihat dari Bumi dengan bentuk cahaya sabit kecil. Bulan baru adalah piringan kecil Bulan yang muncul setelah mengalami satu putaran penuh pada fase Bulan mengelilingi Bumi.
  Kemunculan hilal (bulan baru) merupakan penentuan awal bulan dalam Kalender Hijriah di Indonesia, terkhusus pada bulan Ramadan, Syawal, dan Zulhijah. Kalender merupakan sistem pengorganisasian waktu yang berfungsi sebagai penanda perhitungan dalam jangka panjang. Kalender hijriah termasuk jenis Kalender yang penanggalannya berpatokan pada Bulan ketika mengorbit kepada Bumi.
  Perbedaan antara tahun syamsiah dan tahun kamariah yaitu umur hari dalam satu tahun yang 11 hari juga berbeda dalam penentuan awal perhitungan hari. Penanggalan kamariah memiliki perhitungan yang dimulai sejak terbenamnya Matahari dan berakhir ketika Matahari terbenam pada hari esoknya.

  Sistem penanggalan Islam atau kalender hijriah adalah sistem penanggalan yang memiliki dua belas bulan, dimulai sejak Bulan baru hingga penampakan bulan baru berikutnya berkisar selang waktu antara 29 sampai 30 hari. Revolusi bulan mengelilingi bumi memiliki bentuk lintasan yang elips dengan kecepatan tempuh total dalam satu tahun adalah 354 hari 48 menit dan 34 detik.
  Bulan sebagai salah satu komponen penting dalam penanggalan kamariah yakni merupakan satelit tunggal yang dimiliki Bumi. Bulan memiliki 3 pergerakan, diantaranya pergerakan rotasi atau Bulan berputar pada porosnya, revolusi terhadap bumi dan revolusi bersamaan dengan bumi terhadap matahari.

    \subsubsection{Sejarah Kalender Hijriah}\cite{setyanto2015kriteria}
      Pada saat Sebelum peristiwa haji Wada’ yang dilaksanakan oleh Nabi dan kaum Muslimin, sistem penanggalan masyarakat Arab di Mekah kala itu masih menggunakan konsep penanggalan al-Nasī’. Keberadaan istilah waktu al-Nasī’ tersebut telah mempersulit untuk mengurutkan fenomena/peristiwa yang terjadi sebelum haji Wada’.
    Hal ini dikarenakan aturan penggunaan waktu al-Nasī’ tidak berjalan dengan baik. Bangsa Arab dikenal sering mundur dan memajukan kegiatan-kegiatan yang dilakukan pada bulan-bulan Haram sesuai dengan kebutuhannya. 4 Hal inilah yang menjadikan penanggalan masyarakat Arab sebelum Haji Wada’ dapat dikatakan tidak konsisten.

    Maksud istilah waktu al-Nasī' (waktu pengunduran) yaitu di undur nya waktu untuk melaksanakan suatu kegiatan pada waktu tertentu. Salah satunya adalah pengunduran waktu ibadah haji oleh masyarakat Arah ketika itu. Mereka terkadang melaksanakan ibadah haji pada waktunya, terkadang pula pada bulan Muharam, Ṣafar, dan bulan-bulan lainnya di antara dua belas bulan.
    Dampaknya, adalah hal-hal yang mereka yang biasanya dilakukan pada bulan-bulan haram menjadi terabaikan. Hal ini dikarenakan pada saat mereka sedang melaksanakan ibadah haji, mereka bertemu dengan pembunuh ayah mereka, atau bertemu dengan pembunuh sanak saudara mereka, yang menyebabkan mereka membalas dendam pada waktu tersebut.
    Padahal Allah telah menerangkan bahwa melakukan amalan-amalan saleh pada bulan-bulan tersebut merupakan sebesar-besarnya pahala. Sebaliknya, perbuatan zalim yang dilakukan pada saat itu seburuk-buruknya kesalahan, bahkan menambah kekafiran.
    Namun demikian, konsep al-Nasī’ dimaksudkan untuk menyesuaikan fase Bulan dengan perubahan musim yang diakibatkan oleh posisi dan gerak Matahari di Jazirah Arab. Sehingga dapat dikatakan penanggalan masyarakat Arab ketika itu termasuk menggunakan sistem Penanggalan Matahari-Bulan (Kala Surya-Chandra).
    Meski demikian, Nabi Muhammad beserta umat Islam kala itu mengikuti kalender yang sedang berjalan. Sehingga dapat dikatakan seluruh hidup Nabi Muhammad berpuasa dalam sistem penanggalan yang ditetapkan oleh bangsa Quraisy. Nabi tidak membuat sistem penanggalannya sendiri. Turunnya QS.
    al-Taubah [9]: 36-37, yang melarang penggunaan yaum al-Nasi’ (waktu pengunduran) telah mengubah sistem penanggalan masyarakat Arab dari sistem Lunisolar Calendar menjadi sistem Lunar Kalender. hal Inilah yang menjadi awal mula atau kelahiran sistem penanggalan Islam yang berbasis pada pergerakan Bulan dalam mengelilingi Bumi.
    Hingga saat ini belum diketahui dengan baik bagaimana praktik penanggalan Islam pada zaman sahabat. Namun, diyakini penanggalan Islam pada masa itu didasarkan pada kesaksian ru’yat al-hilāl. Adapun proses bagaimana praktik penanggalan Hijriah sejak berubahnya sistem penanggalan tersebut pada dasarnya dapat ditelusuri melalui sejarah, sebagaimana yang telah dilakukan oleh Saleh al-Saab dari King Abdul’aziz City for Science and Technology (KACST), Riyadh.
    praktik penanggalan Islam kemudian disempurnakan melalui konsep penanggalan yang dirumuskan pada zaman Umar bin Khaṭṭab. Melalui sidang para sahabat rasulullah, ditetapkanlah perhitungan tahun dalam penanggalan kekhalifahan, dimulai sejak hijrah nya Nabi Muhammad dari Mekkah ke Madinah.
    Penetapan tahun hijrah nya Nabi sebagai tahun pertama tersebut merupakan usulan dari Sahabat ‘Ali bin Abī Ṭālib. 11 Oleh karena itu, penanggalan kekhalifahan Islam dikenal sebagai penanggalan Hijriah, dengan bulan Muharam sebagai bulan pertama dalam penanggalan tersebut. Hal tersebut lah yang telah umum berlaku di masyarakat Arab ketika itu.
    Sama halnya dengan penanggalan Masehi yang digunakan saat ini, penanggalan Hijriah pun pada zaman sahabat ditetapkan berdasarkan perhitungan matematis. Jumlah hari yang digunakan senantiasa tetap setiap bulannya. Meskipun demikian, hal-hal yang terkait dengan pelaksanaan ibadah kaum Muslimin kala itu tetap mengikuti ketentuan yang telah diajarkan oleh Nabi Muhammad.
    Oleh karenanya, penanggalan pada kalender Hijriah yang telah ditetapkan merupakan penanggalan Administrasi Negara. Seiring dengan perkembangan pemahaman dan pengetahuan, saat ini fungsi penanggalan Hijriah sebagai penanggalan sosial menjadi satu kesatuan dengan fungsinya sebagai penanggalan ibadah. Hal inilah yang dilihat secara subjektif sebagai kisruh sistem penanggalan Hijriah.
    Maka dari itu, untuk mengurai permasalahan pada tahap awal adalah dengan melepaskan fungsi ibadah dari sistem penanggalan Hijriah.Namun, aturan ibadah tetap menjadi acuan dalam penyusunan kalender Hijriah, sebagaimana yang telah dipraktekkan oleh sahabat. Dalam beribadah terdapat kesepakatan pada proses pencapaian kesatuan dalam beribadah yaitu dapat diawali dengan menyepakati penggunaan kalender tunggal yang digunakan bersama, sedangkan pelaksanaan ibadah dikembalikan kepada masing-masing. Berikut adalah nama bulan dan hari pada kalender Hijriah berdasarkan pada hisab urfi:
    \begin{enumerate}
        \item Muharram      = 30 hari
        \item Shafar        = 29 hari
        \item Rabiul Awwal  = 30 hari
        \item Rabiul Akhir  = 29 hari
        \item Jumadil Awwal = 30 hari
        \item Jumadil Akhir = 29 hari
        \item Rajab         = 30 hari
        \item Shaban        = 29 hari
        \item Ramadhan      = 30 hari
        \item Syawal        = 29 hari
        \item Dzulka'idah   = 30 hari
        \item Dzhulhijjah   = 29/30 hari
    \end{enumerate}


  \subsection{lunisolar calendar/kalender surya candra}
      Menurut Wicaksono dalam artikel nya Lunisolar kalender merupakan sistem kalender candra yang disesuaikan dengan matahari \cite{wicaksono2008ta}.Karena kalender candra dalam 1 tahun mempunyai 11 hari lebih cepat dari kalender surya, maka dalam kalender surya candra memiliki bulan interkalasi (bulan tambahan/bulan ke -13) setiap 3 tahun, agar kembali sesuai dengan perjalanan matahari.
    beberapa contoh kalender yang mengacu pada sistem surya candra adalah kalender imlek/cina, saka, dan Buddha. Semua kalender tersebut tidak ada yang sempurna ,karena jumlah hari dalam satu tahun itu tidak bulat, dan untuk memperkecil eror itu maka dibuat kesepakatan sehari lebih panjang atau terdapat bulan tambahan dalam kalender cina pada tahun kabisat\cite{wicaksono2008ta}.
    Pada kalender surya, pergantian hari terjadi tengah malam dan awal setiap bulan (tanggal 1) yang tidak tergantung pada posisi bulan dan pada kalender candra dan surya candra pergantian hari terjadi ketika matahari terbenam dan awal setiap bulan adalah saat konjungsi (imlek, Sakka, Buddha) atau dalam Hijriah saat munculnya hilal.



\part[Dasar Pemetaan]
{Dasar Dasar\\ Pemetaan}

\chapter[Mengenal Bangun Ruang]
{Dasar Pemetaan\\ Mengenal Bangun Ruang}
\input{chapter/BangunRuang.tex}

\chapter[Mengenal Diagram Kartesius]
{Dasar Pemetaan\\ Kartesius}
\input{chapter/kartesius.tex}

\chapter[Garis Khatulistiwa]
{Dasar Pemetaan\\ Garis Khatulistiwa}
\input{chapter/khatulistiwa.tex}

\chapter[Kordinat Indonesia]
{Dasar Pemetaan\\ Kordinat Indonesia}
\input{chapter/koordinatindo.tex}

\chapter[Kordinat Internasional]
{Dasar Pemetaan\\ Kordinat Internasional}
\input{chapter/longlat.tex}


\part[Data Geospasial]
{Data Geospasial\\ Tipe Data}


\chapter[Data Raster]
{Data Geospasial\\ Data Raster}
% Kelompok 2 Tugas 2 GIS (DATA RASTER)
% Tiara Rizki Wulansari (1154026)
% Muhamad Rifan Zamaludin (1154088)
% Mohammad Agung Deomartha (1154032)
% M. Fajri Mualim (1154078)
% Faisal Syarifuddin (1154104)

\section{Data Raster}
\subsection{Pengertian Data Raster}
Data raster \cite{puntodewo2003sistem} adalah data yang disimpan dalam bentuk persegi empat sama sisi (grid) sel sehingga terbentuk suatu ruang yang 
teratur. Foto digital seperti areal fotografi atau satelit merupakan bagian dari data raster pada peta. 
Raster memiliki data grid continue. Nilainya menggunakan gambar berwarna seperti fotografi, yang ditampilkan dengan 
level merah, hijau, dan biru pada sel. Data Raster (atau disebut juga dengan sel grid) merupakan data yang 
dihasilkan dari sistem penginderaan jauh. Pada data raster. Obyek geografis direpresentasikan sebagai struktur
sel grid yang disebut dengan pixel (picture element). Pada data raster. Resolusi (definisi visual) tergantung
pada ukuran pixelnya. Dengan kata lain. Resolusi pixel menggambarkan ukuran sebenarnya dipermukaan bumi 
yang diwakili oleh setiap pixel pada citra. Pada data raster, Obyek geografis direpresentaskan sebagai struktur sel grid yang disebut sebagi pixel (picture element). Resolusi (definisi visual) tergantung pada ukuran pixel-nya, semakin kecil ukuran permukaan bumi yang direpresentasikan oleh sel, semakin tinggi resolusinya. Data Raster dihasilkan dari sistem penginderaan jauh dan sangat baik untuk merepresentasikan batas-batas yang berubah secara gradual seperti jenis tanah, kelembaban tanah, suhu, dan lain-lain. Peta raster adalah peta yang diperoleh dari fotografi suatu areal. foto satelit atau foto permukaan bumi yang diperoleh dari komputer. Contoh peta raster yang diambil dari satelit cuaca. Di dalam Sig, data raster dan analisis data raster banyak digunakan untuk pemetaan obyek yang bersifat kontinu (batasnya tidak terlihat jelas di lapangan/gradual) dan pemodelan spasial, baik statis maupun dinamis. Analisis data raster banyak menggunakan peta-peta hasil analisa digital citra satelit karena peta raster dan citra satelit mempunyai struktur data yang sama, yaitu grid cell, sehingga kompatibel satu degan yang lain. Hal ini berbeda dengan data vector, dimana agar bisa dianalisis secara bersama, Data raster hasil analisis harus dikonversi dulu ke struktur data vektor \cite{puntodewo2003sistem}.

\subsection{Pengertian Data Vektor}
Data vektor adalah data yang direkam dalam bentuk koordinat titik yang menampilkan, 
menempatkan dan menyimpan data spasial dengan menggunakan titik, 
garis atau area (polygon). Ada tiga tipe data vektor (titik, garis, 
dan polygon) yang bisa digunakan untuk menampilkan informasi pada peta. 
Titik bisa digunakan sebagai lokasi sebuah kota atau posisi tower radio. 
Garis bisa digunakan untuk menunjukkan route suatu perjalanan atau menggambarkan boundary. 
Poligon bisa digunakan untuk menggambarkan sebuah danau atau sebuah Negara pada peta dunia.

\subsection{Kelebihan dan kekurangan Data Raster}
\subsubsection{Kelebihan Data Raster}
Adapun kelebihan yang dimiliki oleh data raster menurut \cite{irwansyah2013sistem} adalah: 
	\begin{enumerate}
		\item memiliki struktur data yang sederhana
		\item mudah dimanipulasi dengan menggunakan fungsi-fungsi matematis sederhana
		\item teknologi yang digunakan cukup murah dan tidak begitu kompleks sehingga pengguna dapat membuat sendiri program aplikasi yang menggunakan citra raster
		\item compatible dengan citra-citra satelit penginderaan jauh dan semua image hasil scanning data spasial.
		\item Overlay dan kombinasi data raster dengan data inderaja mudah dilakukan.
		\item memiliki kemampuan-kemampuan pemodelan dan analisis  spasial tingkat lanjut.
		\item metode untuk mendapatkan citra raster lebih mudah
		\item Gambar permukaan bumi dalam bentuk citra raster yang didapat dari radar atau satelit penginderaan jauh selalu aktual dari pada bentuk vektornya
		\item prosedur untuk memperoleh data dalam bentuk raster lebih mudah, sederhana dan murah.
		\item Harga sistem perangkat lunak aplikasinya cenderung lebih murah.
	\end{enumerate}

\subsubsection{Kekurangan Data Raster}
Adapun Kekurangan yang dimiliki oleh data raster menurut \cite{irwansyah2013sistem} adalah :
	\begin{enumerate}
		\item secara umum memerlukan ruang atau tempat penyimpanan (disk) yang besar dalam komputer, banyak terjadi redudancy data baik untuk setiap layer-nya maupun secara keselururhan.
		\item Pengunaan sel atau ukuran grid yang lebih besar untuk menghemat ruang penyimpaanan akan menyebabkan kehilangan informasi dan ketelitian.
		\item sebuah citra raster hanya mengandung satu tematik saja sehingga sulit digabungkan dengan atribut-atribut lainnya dalam satu layer.
		\item tampilan atau representasi dan akurasi posisi sangat bergantung pada ukuran pikselnya (resolusi spasial)
		\item sering mengalami kesalahan dalam menggambarkan bentuk dan garis batas suatu obyek. sangat bergantung pada resolusi spasial dan toleransi yang diberikan
		\item transformasi koordinat dan proyeksi lebih sult dilakukan
		\item sangat sulit untuk mepresentasikan hubungan topologi (juga network)
		\item metode untuk medapatakan format data vektor melalui proses yang lama, cukup melelahkan dan relatif mahal.
	\end{enumerate}

\subsection{perbedaan data raster dan data vektor}
Masing-masing format data mempunyai kelebihan dan kekurangan.
Pemilihan format data yang digunakan sangat tergantung pada tujuan penggunaan. 
Data yang tersedia, volume data yang dihasilkan, ketelitian yang diinginkan, serta kemudahan dalam Analisa.

Data Vektor relatif lebih ekonomis dalam hal ukuran file dan presisi dalam lokasi. Tetapi sangat sulit untuk 
digunakan dalam komputasi matematik.
Sebaliknya Data raster biasanya membutuhkan ruang pentyimpanan file yang lebih besar dan presisi lokasinya lebih rendah.
Tetapi lebih mudah digunakan secara matematis.
Model data raster mempunyai struktur data yang tersusun dalam bentuk matriks atau pixel dan membentuk grid. 
Setiap pixel memiliki nilai tertentu dan memiliki atribut tersendiri, termasuk nilai kordinat yang unik.

Tingkat keakurasian model ini sangat tergantung pada ukurasn piksel atau biasa disebut resolusi.
Model data ini biasanya digunakan dalam remote sensing yang berbasiskan citra satelit maupun airborne (pesawat terbang).
Selain itu model ini digunakan pula dalam membangun model ketinggian digital(DEM-Digital Elevation Model) dan model permukaan digital(DTM-Digital Terrain Model).
Model Raster Memberikan Informasi spasial terhadap permukaan di bumi dalam bentuk gambaran yang digeneralisasi.
Representasi dunia nyata disajikan sebagai elemen matriks atau piksel yang membentuk grid  yang homogen. 
Pada setiap piksel mewakili setiap obyek yang terekam dan ditandai dengan nilai-nilai tertentu.
Secara konseptual model data raster merupakan model data spasial yang paling sederhana. 

\subsection{karakteristik data raster}
Resolusi suatu data raster akan merujuk pada ukunan permukaan bumi yang direpresentasikan oleh setiap piksel. 
Makin kecil ukuran atau luas permukaan bumi yang dapat direpresentasikan oleh setiap pikselnya, 
makin tinggi resolusi spasialnya.
Piksel-piksel di dalam zone atau area yang sejenis memiliki nilai (isi piksel atau ID number) yang sama. 
Pada umumnya, lokasi di dalam model data raster, diidentifikasi dengan menggunakan pasangan koordinat kolom dan baris (x,y).

\subsection{Metode Penyimpanan Data Raster}
Data raster mempunyai beberapa metode penyimpanan data yaitu run length encoding, block encoding, chain encoding dan quadtree data structure. Dibawah ini terdapat beberapa contoh gambar metode penyimpanan data raster diantaranya :
	\begin{itemize}
		\item  Run Length Encoding(RLE)(Gambar \ref{runlengthencoding} ). Dengan mengurangi jumlah data pada setiap baris. Format RLE, memberikan kelebihan berupa jumlah byte citra yang dapat dimanfaatkan tanpa mengurangi kandungan informasinya. Prinsip penyimpananya ialah dengan mengekspresikan kembali jumlah piksel yang berurutan dengan nilai yang sama sebagai satu pasangan nilai.
				\begin{figure} [ht]
					\centerline{\includegraphics[width=1\textwidth]{figures/runlengthencoding.JPG}}
					\caption{Gambar Run Length Encoding}
					\label{runlengthencoding}
				\end{figure}

		\item  Block Encoding(Gambar \ref{blockencoding} ). Metode ini memperluas dari Run Length encoding menggunakan rangkaian blok untuk menyimpan data. Hampir menyerupai RLE, namun perbedaanya terletak pada dimensionalnya. untuk RLE hanya sepanjang baris saja tetapi block encoding secara dua dimensional (Baris juga Kolom) .
				\begin{figure} [ht]
					\centerline{\includegraphics[width=1\textwidth]{figures/blockencoding.JPG}}
					\caption{Gambar Block Encoding}
					\label{blockencoding}
					
				\end{figure}

		\item  Chain Encoding(Gambar \ref{chainencoding}). Metode pengurangan data dengan mendefinisikan batas-batas entitas. Metode ini mempresentasikan batas (Boundary) suatu region dengan menggunakan rangkaian arah kardinal dan cell-cell, misal N1 berarti berarah ke utara sejauh 1 cell dan S5 berari berarah keselatan sejauh 5 cell.
				\begin{figure} [ht]
					\centerline{\includegraphics[width=1\textwidth]{figures/chainencoding.JPG}}
					\caption{Gambar Chain Encoding}
					\label{chainencoding}
				\end{figure}

		\item  Quadtree Data Structure. Membagi setiap sel dalam image ke dalam empat per empat bagian lalu dibagi lagi ke dalam kelas-kelas. Metode ini menggunakan dekomposisi rekursif dengan membagi grid menjadi hirarki kuadran. Sebuah kuadran yang memiliki nilai yang sama tidak akan dibagi lagi dan disimpan sebagai leaf node. Gambar \ref{quadtreedatastructure} adalah ilustrasi dari Quadtree.
				\begin{figure} [ht]
					\centerline{\includegraphics[width=0.25\textwidth]{figures/quadtreedatastructure.JPG}}
					\caption{Gambar Quadtree Data Structure}
					\label{quadtreedatastructure}
				\end{figure}
	\end{itemize}

\subsection{Akses ikonik ke repositori format data raster data monokrom elektronik jarak jauh}
Dalam Akses ikonik ke repositori format data raster data monokrom elektronik jarak jauh, 
Dokumen disimpan dalam sistem menggunakan monokrom, format raster. 
Dokumen dikirimkan dari repositori ke situs akses jarak jauh untuk ditampilkan kepada pengguna. 
Kemampuan tambahan disediakan untuk mencari dokumen yang tersimpan; 
menghasilkan layar antarmuka pengguna sesuai permintaan yang berisi hasil pencarian; 
memasukkan dokumen ke dalam repositori via transmisi oleh mesin faksimili; 
dan untuk berkomunikasi secara interaktif antara pengguna sistem. 
Dokumen elektronik bisa berupa teks dan grafis konvensional; 
atau dokumen multi media yang berisi teks, video, dan materi audio. 
Sebuah repositori dokumen fisik tunggal dapat secara logis tersegmentasi menjadi
beberapa repositori virtual yang mendukung beragam kelompok pengguna.

\subsection{Pengertian PostGIS}
Sistem Informasi Geografis (SIG) adalah sistem informasi khusus yang mengelola data yang memiliki informasi spasial (bereferensi keruangan). 
Atau dalam arti sempit, adalah sistem komputer yang memiliki kemampuan untuk membangun,
menyimpan, mengelola dan menampilkan informasi bereferensi geografis, misalnya data yang diidentifikasi menurut lokasinya, dalam sebuah database.
SIG juga merupakan sejenis perangkat lunak, perangkat keras (manusia, prosedur, basis data) yang berguna untuk proses pemasukan, penyimpanan, menampilkan data geografis serta atribut-atribut yang terkait.
PostGIS adalah extender database spasial gratis untuk PostgreSQL, 
setiap bit sebaik perangkat lunak berpemilik. Dengan itu, 
Anda dapat dengan mudah membuat query dengan sadar lokasi hanya dalam beberapa baris kode SQL 
dan membangun bagian belakang untuk pemetaan, analisis raster, 
atau aplikasi perutean dengan sedikit usaha. 
PostGIS dalam Tindakan, mengajarkan untuk memecahkan masalah real- 
masalah geodata dunia Ini pertama memberi Anda latar belakang GIS vektor,
raster, dan topologi berbasis GIS dan kemudian dengan cepat bergerak 
untuk menganalisis, melihat, dan memetakan data.





\chapter[Data Vektor]
{Data Geospasial\\ Data Vektor}
\input{chapter/datavektor.tex}

\chapter[Open Geospatial Consortium]
{Standar\\ Open Geospatial Consortium}
\input{chapter/OpenGeospatialConsortium.tex}

\chapter[Web Map Tile Service]
{Data Geospasial\\ Web Map Tile Service}
\input{chapter/wmts.tex}

\chapter[Web Map Service]
{Data Geospasial\\ Web Map Service}
\input{chapter/wms.tex}

\chapter[Data Vektor Line]
{Data Geospasial}
\input{chapter/Tugas2GISKel2D43C.tex}

\chapter[Shapefile]
{Data Geospasial\\ ShapeFile}
\input{chapter/Shapefile.tex}


\chapter[Shapefile Point]
{Data Geospasial\\ Point}
\input{chapter/DefinisiPoint.tex}

\chapter[Shapefile Poligon]
{Data Geospasial\\ Poligon}
\input{chapter/Polygon.tex}


\part[Pemrograman SIG]
{Pemrograman SIG\\ Python}

\chapter[Python]
{Pemrograman SIG\\ Python}
\input{chapter/Python.tex}

\chapter[Variabel]
{Pemrograman SIG\\ Variabel Python}
\input{chapter/2_variabel.tex}


\chapter[Looping]
{Pemrograman SIG\\ Looping Python}
\input{chapter/2_Looping_pada_python.tex}

\chapter[Kelas dan Fungsi di Python]
{Pemrograman SIG\\ Kelas dan Fungsi di Python}
\input{chapter/2_ContohKelasdanFungsipadaPhyton.tex}

\chapter[OpenLayer]
{Pemrograman SIG\\ OpenLayer}
\input{chapter/OpenLayer.tex}

\chapter[LeafletJS]
{Pemrograman SIG\\ LeafletJS}
\input{chapter/leafletjs.tex}


\part[Aplikasi SIG]
{Aplikasi SIG}

\chapter[Mapfile]
{Aplikasi SIG\\ Mapfile}
%M. Amran Hakim Siregar 1154106


\section{Raster Layer dan Vektor Layer}

\subsection{Raster Layer}
Dalam raster layer  setiap lokasi direpresentasikan sebagai suatu posisi sel. 
Sel ini diorganisasikan dalam bentuk kolom dan baris sel-sel dan biasa disebut sebagai grid. 
Dengan kata lain,  layer raster menampilkan, menempatkan, 
dan menyimpan data spasial dengan menggunakan struktur matriks atau piksel-piksel yang membentuk grid. 
Setiap piksel atau sel ini memiliki atribut tersendiri, termasuk koordinatnya yang unik.


\subsection{Vektor Layer}
Data vektor adalah data yang diperoleh dalam bentuk koordinat titik yang menampilkan, 
menempatkan dan menyimpan data spasial dengan menggunakan titik, garis atau area (poligon). 
Terdapat tiga tipe bentuk data vektor (titik, garis, dan poligon) yang bisa digunakan untuk menampilkan informasi pada peta. 
Titik bisa digunakan sebagai lokasi sebuah tempat atau posisi tertentu dalam peta. 
Garis bisa digunakan untuk menunjukkan route suatu perjalanan atau menggambarkan batas suatu wilayah 
dan juga batas suatu kawasan hutan atau area tertentu. 
Poligon bisa digunakan untuk menggambarkan sebuah danau atau sebuah luasan areal yang kemudian dapat analisis luasan 
pada areal-areal tersebut.

Vektor Layer terdiri dari masing-masing titik, yang (untuk data 2D) disimpan sebagai pasangan koordinat (x, y). 
Poin dapat digabungkan dalam urutan tertentu untuk membuat garis, atau bergabung dalam cincin tertutup untuk 
membuat poligon, namun semua data vektor pada dasarnya terdiri dari daftar koordinat yang menentukan simpul, 
beserta peraturan untuk menentukan apakah dan bagaimana simpul tersebut digabungkan. .
Perhatikan bahwa sementara data raster terdiri dari sel array yang teratur, 
titik-titik pada dataset vektor tidak boleh secara teratur spasi.

\subsection{Class}
Class adalah prototype, atau blueprint, atau rancangan yang mendefinisikan variable dan method-methode pada seluruh objek tertentu. 
Class berfungsi untuk menampung isi dari program yang akan di jalankan, di dalamnya berisi atribut / type data dan method untuk menjalankan suatu program.
Class merupakan suatu blueprint atau cetakan untuk menciptakan suatu instant dari  object. class juga merupakan grup suatu object dengan kemiripan attributes/properties, behaviour dan relasi ke object lain. Contoh : Class Person, Vehicle, Tree, Fruit dan lain-lain.

\subsection{STYLE Objects}
Yang dimaksud dengan objek pada java adalah sekumpulan software yang terdiri dari variable dan method-method yang terkait. 
Objek juga merupakan benda nyata yang di buat berdasarkan rancangan yang di definisikan di dalam class

Object adalah instance dari class. Jika class secara umum mepresentasikan (template) sebuah object, 
sebuah instance adalah representasi nyata dari class itu sendiri. Contoh : Dari class Fruit kita dapat membuat object Mangga, 
Pisang, Apel, dan lain-lain.


\chapter[QGIS]
{Aplikasi SIG\\ Desktop QGIS}
\input{chapter/qgis.tex}

\chapter[Python Shapefile]
{Python\\ Shapefile}
%kelompok 4 D4 TI-2D
%Ayu Permata Sari        1154022
%Librantara Erlangga     1154071
%Martin Luter Zega       1154120
%Putri Aulia Ramadhanie  1154096
%Ryan Hafizh Herdiana    1154067
%Copyright (c) 2017 Copyright Holder All Rights Reserved.

\section{Pyshp}
Shapefile adalah file yang berisi domain map dataset dan dapat dibuka dengan aplikasi-aplikasi tertentu yang memiliki fitur GIS didalamnya.
Python adalah bahasa pemrograman yang dapat digunakan untuk membuka shapefile tersebut.
Pyshp adalah library python yang berfungsi agar bisa membaca shapefile, salah satunya format shp.
Pip adalah Package Management System yang berfungsi untuk meng-install dan me-manage paket software yang ada didalam Python.
Metode yang saya lakukan diatas adalah menghitung jumlah shape yang tersedia didalam file shapefile tersebut.

\subsection{ESRI Shapefile}
	
Shapefiles awalnya dikembangkan oleh Environmental Systems Research Institute (ESRI, 1998)\ref{esri}
	\begin{figure}[ht]
	\centerline{\includegraphics[width=1\textwidth]{figures/esri.JPG}}
	\caption{ESRI.}
	\label{esri}
	\end{figure}
	dan merupakan format data geospasial untuk perangkat lunak sistem informasi geografis , seperti ESRI ArcGIS dan Quantum GIS.
	Format shapefile spasial menggambarkan geometri sebagai fitur titik, polyline, atau poligon

Shapefile adalah pengelompokan beberapa file untuk mewakili aspek geodata yang berbeda:
\begin{enumerate}
	\item .shp: bentuk format; geometri fitur itu sendiri.
	\item .shx: bentuk format indeks; indeks posisi geometri fitur untuk memungkinkan pencarian cepat.
	\item .dbf: format atribut; atribut columnar untuk setiap bentuk, dalam format dBase IV.
\end{enumerate}
Atribut fitur dapat diperiksa di perangkat lunak SIG, atau dengan membuka berkas .dbf di Microsoft Excel, misalnya. Ada juga beberapa file pilihan dalam format shapefile. Yang paling penting adalah file .prj yang menggambarkan sistem koordinat dan informasi proyeksi yang digunakan. Shapefiles yang dihasilkan untuk model bahaya NBCC2015 ditentukan dengan proyeksi WGS84

\subsection{Instalasi pyshp}
Install Python terlebih dahulu
Install PIP di python
Buka cmd lalu ketik: Python m pip install pyshp ,lalu enter
Kemudian upgrade pip dengan mengetikkan di cmd: Python m pip install upgrade pip ,lalu enter
Lalu masuk ke python

\subsection{Dasar Penggunaan}
berikut ini adalah dasar penggunaan module pyshp di python

Membuat file pada pyshp
  \begin{verbatim}
  import shapefile
  a = shapefile.Writer(shapeType=1)
  a.field('nama','C','40')
  a.field('alamat','C','255')
  a.save('namafile.shp')
  \end{verbatim}

Mengedit atau Menambah record
\begin{verbatim}
  import shapefile
  a = shapefile.Editor(shapefile='namafile.shp')
  a.record('politeknis pos','jl.sarijadi')
  a.point(-6.8731953,107,5737873,0,0)
  a.save('namafile')
\end{verbatim}

Menghapus record
\begin{verbatim}
  import shapefile
  a = shapefile.Editor('namafile.shp')
  a.shape(0) //masukan data ke berapa //karena array jadi dimulai dari 0
  a.delete(0)
  a.save('namafile')
\end{verbatim}

Membaca record
\begin{verbatim}
  import shapefile
  a = shapefile.Reader('namafile.shp')
  a.records() //menampilkan semua record
  a.record(0) //spesifik record
\end{verbatim}












\bibliographystyle{IEEEtran}
\bibliography{koordinatindo,definisi,longlat,BangunRuang,kartesius,benua,khatulistiwa,kutubutara,sejarahwaktu,Antartika,penanggalan,ptolemy,aliddrissi,willemblaeu,Shapefile,referensikel2,reference,references,python,qgis,DataRaster,datavektor,wmts,mapserver,datavektor,aliddrissi}


\printindex
\end{document}

