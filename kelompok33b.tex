% Kelompok 3
% Indah Rahmawati (1154070)
% Boby Jamis Hari Sel (1154040)
% Dhea Amelia (1154123)



\section{Pendahuluan Mapfile}
    Map  File  merupakan  sebuah  konfigurasi  data  yang terstruktur  yang  digunakan  pada  MapServer.  Map  File mendefinisikan  area-area  pada  peta  dan  juga  menentukan input-an dan output-an kepada program MapServer. Selain itu Map File juga berfungsi untuk mendefinisikan lapisan layer yang digunakan pada MapServer yang meliputi data atribut, proyeksi peta, dan simbol-simbol. Map File harus memiliki format .Map untuk dapat diidentifikasi oleh MapServer.MapScript yang digunakan sebagai interface untuk class-class yang terdapat di MapServer.Secara garis besar,peta tersusun dari beberapa layer.Layer tersusun dari bentuk-bentuk yang polygon,garis, atau titik yang disebut dengan Shape.Class-class yang terdapat di MapServer melingkupi manipulasi untuk Peta,Layer,dan Shape.Class-class didalam MapServer yang sering digunakan untuk mengembangkan Web GIS.Struktur map file yang digunakan berisi konfigurasi dari peta yang akan ditampilkan isinya memuat beberapa bagian penting yang menjadi kelas utama untuk diproses,bagian-bagian ini disebut dengan parameter yang nantinya akan diproses oleh map script obyek untuk penanganan obyek yang akan disajikan dalam sebuah website.

\section{Pengertian Class}
    Class merupakan sesuatu yang berhubungan dengan MapFile atau secara tidak langsung berhubungan dengan data peta.Class ini juga sebagai class utama di MapScript untuk memproses dan menyimpan data peta ke file gambar. 
    
\subsection {Fungsi- Fungsi Class}
Beberapa fungsi penting yang terdapat pada class,antara lain :
\begin{enumerate}
\item Mempunyai kumpulan class layer Object yang menyusun peta dan fungsi untuk mengatur urutan peta.
\item Untuk menggambar peta,disimpan ke dalam class image Object.
\item Untuk menggambarkan legend,dan disimpan ke dalam classs image object.
\item Untuk menggambar scalebar, dan disimpan ke dalam class image object.
\item Untuk mengatur kelaskelas pada sebuah obyek(layer).
\item Fungsi SetExtent untuk menentukan Extent dari peta.
\item Fungsi ZoomPoint, ZoomRectangle,ZoomScale untuk melakukan pembesaran (zoom in) atau pengecila (zoom out peta).
\item Untuk mengeset proyeksi peta.
\item QueryByPoint, QueryByRect,QueryByShape,QueryByFeature untuk mencari object di peta yang ada di posisi tertentu dengan Rectangle, shhape, dan dengan peta lain.
\item Fungs GetShape untuk mengambil sebuah shape yang ada pada layer.Disimpan dalam shapeObj class.
\item AddFeature untuk menambah layer dengan sebuah shape baru.

\section{Class Object}
    semua fitur dalam seperangkat data spasial yang diberikan dalam lapisan harus memiliki tipe geometrika yang sama. Namun, tidak setiap titik, garis, atau poligon dalam kumpulan data perlu memiliki makna yang sama dalam arti kartografi sehingga peta harus membedakan jalan dari jalur sepeda untuk menghindari kebingungan (mungkin bencana), misalnya.

\section {Pendahuluan Symbols}
    Peta adalah representasi abstrak yang memanfaatkan simbol titik, garis dan area. Bertin (1974) menciptakan skema simbol yang jelas dan logis dimana simbol dapat bervariasi mengacu pada variabel grafis. Variabel grafis berikut dapat digunakan di MapServer: FORMULIR, UKURAN, POLA, WARNA DAN LIGHTNESS. Simbol titik dan area serta font teks (ttf) juga dapat ditampilkan dengan bingkai yang kita sebut GARIS. Simbol tidak mengandung informasi warna, warna diatur dalam objek style. Jika STYLE menggunakan simbol ini tidak mengandung ukuran yang eksplisit, maka ukuran simbol default akan didasarkan pada kisaran nilai "y" pada koordinat titik. misalnya jika koordinat y dari titik-titik di kisaran simbol dari 0 sampai 5, maka ukuran default untuk simbol ini akan diasumsikan 5.
    
\subsection {Simbol Scaling}
Ada dua cara yang berbeda untuk menangani ukuran layar simbol dan elemen kartografi di peta dengan skala yang berbeda. Ukuran elemen kartografi diatur sebagai berikut :
\begin{enumerate}
\item Jika ukurannya diatur dalam unit dunia nyata (misalnya meter), simbol akan menyusut dan tumbuh sesuai skala tampilan peta.
\item Jika ukurannya diatur dalam piksel layar, simbol terlihat sama pada semua skala.

Perilaku default MapServer adalah menerapkan tipe layar "layar piksel" untuk menampilkan elemen kartografi.

\subsection {Spesifikasi MapServer dan simbol}
Dalam aplikasi MapServer, parameter SYMBOL diatur dalam file peta sebagai berikut:
\begin{enumerate}
\item Setiap LAPIS memiliki parameter TYPE yang mendefinisikan tipe geometri (titik, garis atau poligon). Simbol diberikan pada titik, sepanjang garis atau area yang sesuai.
\item Simbol dasar didefinisikan dalam elemen SYMBOL , dengan menggunakan parameter TYPE , POINTS , IMAGE , FILLED , ANCHORPOINT dan lainnya (elemen SYMBOL dapat dikumpulkan dalam file simbol terpisah untuk digunakan kembali).
\item Warna, ringan, ukuran dan garis besar didefinisikan di dalam bagian GAYA dari bagian KELAS menggunakan parameter COLOR , SIZE , WIDTH dan OUTLINECOLOR .
\item Pola untuk garis styling dan poligon didefinisikan pada bagian GAYA dengan GAP dan POLA .
\item Beberapa elemen dasar dapat dikombinasikan untuk mencapai tanda tangan yang kompleks dengan menggunakan beberapa STYLE s dalam satu KELAS .

\subsection {Definisi Symbols}
    simbol dapat disertakan dalam file peta utama atau lebih umum lagi dalam file terpisah. Dalam simbol terdapat file terpisah ditunjuk menggunakan kata kunci SYMBOLSET , sebagai bagian dari objek MAP . Pengaturan yang disarankan ini sangat ideal untuk menggunakan kembali definisi simbol pada beberapa aplikasi MapServer.Simbol digunakan untuk bekerja dengan data peta lebih detail terutama untuk menangani penggunaan obyek simbol pada peta.

\subsection {Jenis-jenis symbols}
Ada 3 jenis simbol utama di MapServer: 
\begin{enumerate}
\item Marker
\item Garis 
\item Shadesets.

\section {Pengetian Label}
Label merupakan class yang dapat mengatur label-label yang akan tampil pada peta.Untuk melengkapi informasi pada peta,label perlu diberikan berbagai macam keterangan-keterangan. Tambahan informasi tersebut berupa atribut peta yang tersedia pada template seperti nama pembuat, tahun pembuatan, nama-nama tempat pada dan di sektar lokasi peta, dan lain-lain.Label juga dapat digunakan untuk membersihkan memori yang dialokasikan sebagai cache label dari peta yang akan ditampilkan.
