% Kelompok 3
% Indah Rahmawati (1154070)
% Boby Jamis Hari Sel (1154040)
% Dhea Amelia (1154123)



\section{Pendahuluan Mapfile}
    Map  File  merupakan  sebuah  konfigurasi  data  yang terstruktur  yang  digunakan  pada  MapServer.  Map  File mendefinisikan  area-area  pada  peta  dan  juga  menentukan input-an dan output-an kepada program MapServer. Selain itu Map File juga berfungsi untuk mendefinisikan lapisan layer yang digunakan pada MapServer yang meliputi data atribut, proyeksi peta, dan simbol-simbol. Map File harus memiliki format .Map untuk dapat diidentifikasi oleh MapServer.

\section{Pengertian Class}
    Class merupakan sesuatu yang berhubungan dengan MapFile atau secara tidak langsung berhubungan dengan data peta.Class ini juga sebagai class utama di MapScript untuk memproses dan menyimpan data peta ke file gambar.
    
\subsection {Fungsi- Fungsi Class}
Beberapa fungsi penting yang terdapat pada class,antara lain :
\begin{enumerate}
\item Mempunyai kumpulan class layer Object yang menyusun peta dan fungsi untuk mengatur urutan peta.
\item Untuk menggambar peta,disimpan ke dalam class image Object.
\item	Untuk menggambarkan legend,dan disimpan ke dalam classs image object.
\item	Untuk menggambar scalebar, dan disimpan ke dalam class image object.
\item   fungsi yang digunakan untuk mengatur kelaskelas pada sebuah obyek(layer).

\section{Pendahuluan Symbols}
    Peta adalah representasi abstrak yang memanfaatkan simbol titik, garis dan area. Bertin (1974) menciptakan skema simbol yang jelas dan logis dimana simbol dapat bervariasi mengacu pada variabel grafis. Variabel grafis berikut dapat digunakan di MapServer: FORMULIR, UKURAN, POLA, WARNA DAN LIGHTNESS. Simbol titik dan area serta font teks (ttf) juga dapat ditampilkan dengan bingkai yang kita sebut GARIS. Simbol tidak mengandung informasi warna, warna diatur dalam objek style. Jika STYLE menggunakan simbol ini tidak mengandung ukuran yang eksplisit, maka ukuran simbol default akan didasarkan pada kisaran nilai "y" pada koordinat titik. misalnya jika koordinat y dari titik-titik di kisaran simbol dari 0 sampai 5, maka ukuran default untuk simbol ini akan diasumsikan 5.
    
\subsection {Simbol Scaling}
Ada dua cara yang berbeda untuk menangani ukuran layar simbol dan elemen kartografi di peta dengan skala yang berbeda. Ukuran elemen kartografi diatur sebagai berikut :
\begin{enumerate}
\item Jika ukurannya diatur dalam unit dunia nyata (misalnya meter), simbol akan menyusut dan tumbuh sesuai skala tampilan peta.
\item Jika ukurannya diatur dalam piksel layar, simbol terlihat sama pada semua skala.

Perilaku default MapServer adalah menerapkan tipe layar "layar piksel" untuk menampilkan elemen kartografi.

\subsection {Definisi Symbols}
    Definisi simbol dapat disertakan dalam file peta utama atau lebih umum lagi dalam file terpisah. Definisi simbol dalam file terpisah ditunjuk menggunakan kata kunci SYMBOLSET , sebagai bagian dari objek MAP . Pengaturan yang disarankan ini sangat ideal untuk menggunakan kembali definisi simbol pada beberapa aplikasi MapServer.

\subsection {Jenis-jenis symbols}
Ada 3 jenis simbol utama di MapServer: 
\begin{enumerate}
\item Marker
\item Garis 
\item Shadesets.
