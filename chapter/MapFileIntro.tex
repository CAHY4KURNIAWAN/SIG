\section{Introduction}

Untuk memanfaatkan fungsionalitasnya, MapServer akan membutuhkan sebuah mapfile. Mapfile merupakan file teks yang berekstensi .map,
yang akan mendeskripsikan apa dan dimana sumber data dan bagaimana cara data tersebut ditampilkan. 
File Peta adalah text konfigurasi yang terdiri dari list setting yang digunakan untuk menggambar dan berinteraksi dengan peta. 
Informasi yang termuat didalamnya adalah layer data apa yang akan digambar, dimana focus geografis petanya, 
system proyeksi yang digunakan, format apa yang akan digunakan untuk menampilkan gambar, dan cara menentukan legenda dan skala pada peta.

contoh :
\begin{verbatim}
MAP
    NAME "Bandung"
    STATUS ON
    SIZE 600 400
    SYMBOLSET "../etc/symbols.txt"
    EXTENT 107.474 -7.007 107.735 -6.773
    UNITS DD
    SHAPEPATH "../bandung"
    IMAGECOLOR 255 255 255
    FONTSET "../etc/fonts.txt"

    #
    # Start of web interface definition
    #
    WEB
        IMAGEPATH "/ms4w/tmp/ms_tmp/"
        IMAGEURL "/ms_tmp/"
    END # WEB

    #
    # Start of layer definitions
    #
    LAYER
        NAME 'Bandung_Polygon'
        TYPE POLYGON
        STATUS DEFAULT
        DATA band.png
    END # LAYER
END # MAP
\end{verbatim}

\subsection{Komponen-komponen Utama Mapserver}
Mapserver menghasilkan keluaran berupa file graphic berdasarkan masukan yang diberikan oleh user. Komponen kuncinya adalah MapServer executable yang terdiri dari CGI program, file peta, sumber data dan output gambar. Semua komponen mapserver ini bekerja bersama-sama, setelah user melakukan request/perminataan maka CGI akan mengakses file peta, menggambarkan informasi yang didapat dari sumber data dan kembali menampilkannya pada peta.
	\subsubsection {Mapserver Executable}
		Secara sederhana MapServer menjalankan executable applikasi CGI pada web server yang secara teknis merupakan proses
 		stateless berbasis pada HTTP. Stateless adalah sebuah proses permintaan yang dilanjutkan dengan stop running. Applikasi CGI 		
		menerima permintaan dari web server, kemudian proses dilakukan dan mengembalikan respon atau data ke web server. CGI bekerja 		
		sangat sederhana tidak diperlukan sebuah pemrograman untuk dapat menggunakannya. Kita tinggal melakukan edit berdasarkan text base,
		konfigurasi runtime file, membuat halaman web, dan menempatkannya bekerja pada web server.

\subsection{Cara Kerja Mapserver}
Map Server bekerja secara berdampingan dengan applikasi web server. Web Server menerima request peta melalui MapServer. 
MapServer mengenerate request terhadap peta dan mengirimkannya ke web server,
Fungsi utama dari MapServer adalah melakukan pembacaan data dari banyak sumber dan menempatkannya kedalam layer-layer secara bersamaan menjadi file graphic.
Salah satu layernya bisa saja berupa gambar satelit. Setiap layer saling overlay satu dengan lainnya dan ditampilkan kedalam web browser.
Proses pengambaran peta (rendering) muncul setiap kali permintaan terhadap peta baru yang dilakukan oleh Mapserver termasuk ketika user melakukan level zoom terhadap tampilan peta.

\subsection{Tentang MapFile}
Map File adalah sebuah file konfigurasi teks terstruktur untuk aplikasi MapServer kita. Ini dapat mendefinisikan area peta yang kita buat, 
memberi tahu program MapServer tempat data kita berada dan di mana menampilkan gambar. 
Map File ini juga mendefinisikan lapisan peta kita, termasuk sumber data, proyeksi, dan simbologinya. 
Map File ini harus memiliki ekstensi peta atau MapServer tidak akan mengenalinya.
File .map juga bisa digunakan untuk.
\begin{itemize}
	\item Debugging maps. ini adalah tipe dari plain text atau teks biasa yang menunjukan offset fungsi relatif untuk versi biner tertentu.
	\item Uncompiled Quake maps. File-file ini adalah file teks biasa.
	\item Valve's Hammer Editor menyimpan file tingkat dalam format biner, proprietary .rmf atau text-based, yang dapat dibaca manusia. Sebelum versi 4.0.
	\item Selain itu, serial Halo menggunakan file peta untuk level. Duke Nukem 3D menggunakan file peta untuk level buatan pengguna.
	\item .map file juga digunakan oleh MapInfo Professional Geographic Information System.
	\item .MAPs juga diidentifikasi sebagai peta warna. Contoh perangkat lunak semacam itu yang mendukung file peta adalah visualisasi musik G-Force SoundSpectrum.
	\item Bentuk lain dari file .MAP adalah untuk peta gambar HTML. Peta gambar diformat dalam HTML dan membuat area yang dapat diklik pada gambar yang disediakan.
	\item .map file juga digunakan oleh Multi Theft Auto: San Andreas, modifikasi multiplayer untuk Grand Theft Auto: San Andreas.

\end{itemize}


\subsection{ketentuan penulisan mapfile}
File yang disebut mapfile tersebut memiliki ketentuan-ketentuan penulisan sebagai berikut:

\begin{itemize}
	\item Teks pada mapfile tidak bersifat case-sensitive.
	\item Penulisan string yang berisi campuran beberapa karakter non-alphanumerik (selain karakter huruf dan angka) atau keywords milik MapServer harus diapit oleh tanda petik ganda .
	\item Setiap mapfile dapat digunakan untuk mendefinisikan (secara default) maksimal 50 layer peta.
	\item Penulisan path lokasi file bisa dilakukan secara absolut maupun relatif.
	\item Isi mapfile memiliki hierarki struktur dengan objek “MAP” sebagai “root”. Sementara objek-objek lainnya berada di bawah objek ini.
	\item Pengguna dapat menambahkan baris-baris komentar di dalam mapfile dengan cara mengawali komentar tersebut dengan karakter pagar.
	\item Atribut (field) ditulis didalam kurung siku [] (case sensitive)
	\item Dimulai dengan keyword MAP sebagai puncak hierarki
	\item Untuk MapServer sebelum versi 5, ada default maksimum 200 layer per mapfile, 
	      sedangkan untuk MapServer versi 5 atau lebih tidak ada batasan layer. 
              Hal ini dapat diubah dengan mengedit file map.h untuk merubah value dari MSMAXLAYERS menjadi ke jumlah yang diinginkan.
	\item file path dapat diberikan sebagai path yang absolut, atau sebagai path yang relatif terhadap lokasi mapfile. Selain itu, file data mungkin ditentukan relatif terhadap SHAPEPATH.
	\item Mapfile diharapkan bisa dikodekan UTF-8. Mapfile yang dikodekan UTF-8 tidak perlu dikodekan ke UTF-8. (pada versi 7)
	\item Atribut diberi nama dengan menggunakan sintaks berikut: [ATTRIBUTENAME].			  
	\end{itemize}

\subsection{Hirarki Mapfile}
``MAP`` merupakan ``root``, diikuti objek lain, setiap blok harus diakhiri dengan ``END``, 
contoh:
\begin{verbatim}
MAP
	LAYER
		....
	END
END
\end{verbatim}


\section{MAP Object}
\subsection{Tentang MAP Object}
MAP merupakan objek master sebagai wadah menyimpan semua objek lain didalamnya dan tempat mendefinisikan objek dan parameter 
peta/aplikasi seperti, config, datapattern, debug, status, units, size, extent, fontset, imagecolor, layer, legend, dll.
EXTENT adalah output batas dalam unit dari output peta.
SIZE adalah  lebar dan tinggi peta gambar dalam piksel.
IMAGECOLOR adalah gambar warna latar belakang.

\subsubsection (ANGLE double)
Angle diberikan dalam derajat , untuk memutar peta .Default adalah 0 .Peta akan memutar yang diberikan searah jarum jam .Berikut ini adalah catatan penting:
\begin{itemize}
	\item Membutuhkan sebuah PROJECTION OBJECT dispesifikasikan di MAP LEVEL dan untuk setiap LAYER OBJECT ( bahkan jika semua lapisan yang di proyeksi ) yang sama.
	\item Membutuhkan MapScript (SWIG, PHP MapScript) tidak bekerja dengan CGI mode.
	\item Jika dengan menggunakan LABEL OBJECT atau LAYER OBJECT LABELANGLEITEM parameter tersebut baik, parameter ini yang relatif terhapat peta orientasi.
\end {itemize}

\subsubsection (CONFIG [key] [value])
Hal ini dapat digunakan untuk menentukan nilai di beberapa run-time, untuk kedua mapserver dan GDAL atau OGR libraries. Pengembang: nilai-nilai akan di sampaikan CPLSetConfigOption(). Pilihan berikut tersedia khusus untuk mapserver.
\begin {itemize}
	\item CGI_CONTEXT_URL [value] yaitu parameter konfigurasinya dapat digunakan untuk memungkinkan memuat peta konteks dari sebuah URL.
	\item MS_ENCRYPTION_KEY [filename] yaitu parameter konfigurasinya dapat digunakan untuk menspesifikasikan sebuah kunci enkripsi yang digunakan dengan mapserver.

\subsection{contoh}
	\begin{verbatim}
	MAP
  	NAME "GOR Yayasan Taruna Bakti"
  	EXTENT -6.898614, 107.629599 # Geographic
  	SIZE 800 400
  	IMAGECOLOR 128 128 255
	END # MAP
	\end{verbatim}

	\begin{verbatim}
	MAP
  	NAME "Taman Pustaka Bunga"
  	EXTENT -6.903290, 107.622925 # Geographic
  	SIZE 800 400
  	IMAGECOLOR 128 128 255
	END # MAP
	\end{verbatim}
\section{Layer Object}
Sering digunakan ini mendefinisikan layer-layer yang kemudian membentuk tampilan peta dijitalnya. Layer-layer digambarkan sesuai dengan urutan kemunculannya
\(penulisan\) di dalam mapfile terkait. Dengan demikian, layer pertama akan diletakkan di dasar sehingga tampilannya bisa tertutupi oleh tampilan layer 
berikutnya. Sementara itu, layer yang terakhir \(ditulis paling bawah\) akan muncul paling atas dengan menutupi layer-layer yang berada di bawahnya.

\subsection{CLASS}
CLASS adalah Sinyal awal dari kelas object. Di dalam lapisan, hanya dengan satu kelas aka digunakan untuk render fitur.  
Setiap fitur diuji untuk setiap kelas dalam rangka dimana mereka didefinisikan dalam mapfile.  
Kelas pertama yang tidak kalah dengan bi min / max skala ikatan dan ekspresi memeriksa untuk arus fitur akan digunakan untuk render.  
Class terbagi antara dua yaitu :

  \subsubsection{CLASSGROUP atau string}
  Tentukan kelas group yang akan dipertimbangkan dalam render waktu. Object classgroup parameter harus digunakan dalam kombinasi dengan classgroup.
  \subsubsection{CLASSITEM atau attribute}
  Item nama dalam atribut tabel yang akan digunakan untuk kelas lookups .
  \subsubsection{CLUSTER}
  Sinyal awal dari cluster objek. Konfigurasi cluster menyediakan untuk menggabungkan beberapa fitur dari lapisan menjadi satu ( dikumpulkan ) fitur  hanya berdasarkan relatif posisi. Hanya didukung untuk point layers.
  \subsubsection{COMPOSITE}
  Satu atau lebih blok composite dapat digunakan untuk memberi sinyal bahwa rendering harus dilakukan sementara dan digabungkan ke gambar peta akhir. 
  Opsi yang didefinisikan di dalam blok composit akan menentukan bagaimana penggabungan ini harus dilakukan (mis. Dengan menerapkan opacity, composition operator, atau filter piksel).
  

