% kelompok definisi
% ariana setiawan (1154042)
% idang mawardi (1154084)
% arya niken manalu (1154080)
% M. Arya Sikumbang (1154075)
% r rifa fauzi komara (1154089)
% Andi Tenri Wali (1154013)

\section{Definisi GIS (GEOGRAPHICS INFORMATION SYSTEM)}
Geographical information system (GIS) adalah sebuah komputer yang berbasis sistem
Informasi digunakan untuk memberikan informasi bentuk digital dan analisis terhadap 
Permukaan geografi bumi.

\subsection{Pemahaman pada Geographic Information System GIS}
Dimana GIS merupakan pemahaman dari, sebagai berikut:
\begin{enumerate}
\item Geography


Dimana GIS dibangun berdasarkan pada istilah‘geografi’ atau ‘spasial’.
Object mengacu pada spesifikasi lokasi dalam suatu tempat/ruang. Objek dapat berupa fisik,
budaya ataupun ekonomi alamiah. Penampakan yang seperti ini ditampilkan pada suatu peta yang 
digunakan untuk memberikan gambaran yang lebih representatif dari spasial dari suatu objek.
sesuai dengan kenyataannya yang terdapat di bumi. Dimana simbol, warna dan gaya garis digunakan sebagai
perwakilan dari setiap spasial yang berbeda pada peta dua dimensi.
Pada gambar \ref{dataspasial} dijelaskan bahwa data spasial berikut berupa 
titik, garis, poligon (2-D) dan permukaan (3-D).


\begin{figure}[ht]
	\centerline{\includegraphics[width=1\textwidth]{figures/dataspasial.JPEG}}
	\caption{data spasial berikut berupa titik, garis, poligon (2-D), permukaan (3-D).}
	\label{dataspasial}
	\end{figure}


Dan arti dari gambar di atas adalah:
\begin{itemize} 
\item Format Titik
\end{itemize}

\begin{itemize} 						
\item[--] Memiliki koordinat tunggal 		
\item[--] Tanpa memiliki panjang 			
\item[--] Tanpa memiliki luasan
\end{itemize}

\begin{itemize} 
\item Format Garis
\end{itemize}

\begin{itemize}
\item[--] Memiliki koordinat titik awal dan akhir		
\item[--] Memiliki panjang tanpa luasan
\end{itemize}

\begin{itemize} 
\item Format Polygon
\end{itemize}

\begin{itemize} 					
\item[--] Memiliki koordinat titik awal dan akhir
\item[--] Memiliki panjang dan luasan
\end{itemize} 

\begin{itemize} 
\item Format Permukaan
\end{itemize}

\begin{itemize}
\item[--] Memiliki area koordinat vertikal
\item[--] Memiliki area dengan ketinggian
\end{itemize}


\item Information

Informasi berasal dari kata pengolahan sejumlah data. Di dalam GIS informasi mempunyai
Volume terbesar. Dan setiap object geografi memiliki setting datanya tersendiri karena 
tidak sepenuhnya data yang ada dapat terwakili di dalam peta. Maka, semua data harus
Diasosiasikan pada objek spasial yang mampu membuat peta menjadi intelligent.


\item System

Pengertian dari suatu sistem merupakan kumpulan elemen-elemen yang saling berintegrasi 
Dan berinterdependensi dalam sebuah lingkungan yang dinamis untuk mencapai tujuan tertentu.
\end{enumerate}

\subsection{Definisi GIS (Geography Information and System)}
Dan definisi dari GIS dapat selalu berubah karena GIS adalah bidang kajian ilmu 
Dan teknologi yang masih baru. Beberapa definisi dari Geographical Information System yaitu:
\begin{enumerate}
\item Definisi GIS menurut (Rhind, 1988):
Yaitu: GIS is a computer system for collecting, checking, integrating and analyzing
Information related to the surface of the earth.

\item Definisi GIS menurut (Marble \& Peuquet, 1983) and (Parker,
1988; Ozemoy et al., 1981; Burrough, 1986):
Yaitu: GIS deals with space-time data and often but not necessarily, employs computer
Hardware and Software.

\item Definisi GIS menurut (Purwadhi, 1994):
- SIG adalah suatu sistem yang mampu mengorganisir perangkat keras (hardware),
Perangkat lunak (software), dan data, serta dapat mendayai dan digunakan sistem
Penyimpanan, pengolahan, maupun analisis data yang dilakukan secara simultan, sehingga dapat
Diperoleh seluruh informasi yang berkaitan secara langsung dengan aspek keruangan.
- SIG adalah manajemen data spasial dan data non-spasial yang berbasis komputer
Dengan menggunakan tiga karakteristik dasar, yaitu: 
\begin{itemize}
    \item Memiliki fenomena yang aktual (variabel data non-lokasi) dan berhubungan dengan topik permasalahan di lokasi bersangkutan.
    \item Merupakan suatu kejadian di suatu lokasi tertentu.
    \item Memiliki dimensi waktu. Alasan GIS dibutuhkan adalah karena untuk data spasial.
\end{itemize}

Penanganannya sangat sulit karena peta dan data statistik cepat mengalami kedaluwarsa 
Sehingga tidak ada pelayanan penyediaan data dan informasi yang diberikan menjadi tidak akurat.
\end{enumerate} 

Berikut merupakan keistimewaan analisis dengan Geographical Information System (GIS) yaitu:
\begin{enumerate}
\item Analisis Proximity
Analisis Proximity adalah geografi yang berbasis pada jarak antar layer.
Di dalam analisis proximity GIS menggunakan proses yang disebut dengan buffering
Yaitu membangun lapisan pendukung sekitar layer dalam jarak tertentu agar dapat menentukan
dekatnya hubungan antara sifat bagian yang ada.
\item Analisis Overlay
Analisis Overlay adalah proses integrasi data dari lapisan-lapisan layer yang berbeda (overlay).
Yang secara analisis membutuhkan lebih dari satu layer yang akan di tumpang susun secara
fisik agar dapat di analisis secara visual.
\end{enumerate}

Maka artikel:
	Dalam sebuah artikel dari husein yang menyebutkan bahwa  GIS merupakan pemahaman dari
	Geography, Information dan System \cite{husein2006konsep}.

\section{Geographic Information System (GIS): Introduction to the computer perspective}
Sistem Informasi Geografi (GIS) diartikan sebagai sistem untuk menyimpan, memeriksa, 
mengintegrasikan, memanipulasi, menganalisis dan memaparkan data yang berkaitan dengan semua 
ruang yang berhubungan dengan keadaan bumi.

Maka artikel :
	Dalam sebuah artikel dari prahasta yang menyebutkan bahwa  GIS merupakan menyimpan, memeriksa, mengintegrasi, memanipulasi, menganalisis dan memaparkan data yang berkaitan dengan semua ruang yang berhubungan dengan keadaan bumi\cite{prahasta2009sistem}.

\subsection{Pengenalan GIS atau Geography Information System}
\begin{enumerate}
	
\item GIS atau dikenal dengan Sistem Informasi Geografi ditujukan sebagai sistem yang mampu menyimpan, memeriksa, mengintegrasikan, memanipulasi, menganalisis dan memaparkan data-data yang terkait dengan spasial yang merujuk terhadap bagian bumi. (Jabatan Alam Sekitar, 1987).

\item GIS merupakan satu set lat untuk mengumpulkan, menyimpan, mendapatkan, mengubah dan memaparkan data ruang dari keadaan  bumi yang sebenarnya untuk keperluan tertentu (Burrough, 1986).


\item GIS adalah setiap set manual atau prosedur komputer yang digunakan untuk menyimpan dan memanipulasi data geografis yang tersedia (Arronoff, 1989).


\item GIS merangkum keadaan bumi dengan peranti atau perangkat tertentu yang digunakan untuk peta input atau peta produk, bersama-sama dengan dengan sistem komunikasi yang diperlukan untuk dijadikan sebagai penghubung berbagai unsur. (Star \& Ester, 1990).

\item GIS adalah suatu sistem untuk membantu dalam membangunkan model tertentu yang mustahil untuk dijadikan sintesi data yang banyak. (Martin, 1996).

\end{enumerate}

\subsection{Komponen GIS atau Geography Information System}
Komponen GIS sendiri dibagikan menjadi 3 komponen, yaitu:
Sistem Komputer (perkakas dan sistem operasi), Software GIS

(ArcGIS), database GIS, metode GIS (Prosedur analisis), People (Orang-orang yang menggunakan GIS/User).
Pada gambar \ref{komponen GIS} dijelaskan bahwa komponen GIS sebagai berikut.


\begin{figure}[ht]
	\centerline{\includegraphics[width=1\textwidth]{figures/komponenGIS.JPG}}
	\caption{komponen GIS.}
	\label{komponen GIS}
	\end{figure}

\subsubsection{Komponen GIS atau Geography Information System}
Sesuai dengan gambar diatas komponen GIS dibagi menjadi 3 bagian, yaitu :
\begin{itemize}
    \item Sistem Komputer (perkakas dan sistem operasi), merupakan hardware dari sebuah sistem GIS. Perkakas terdiri dari monitor, unit sistem atau CPU, keyboard dan mouse (Heywood et al., 2002). Teknologi komputer harus memiliki kemampuan kuasa yang tinggi untuk menjalankan per isian GIS.
    \item Software GIS, merupakan ArcGIS untuk tujuan perancangan, pengurusan ataupun pemodelan pada kebutuhan tertentu.
    \item Database GIS, merupakan tempat yang melibatkan data GIS baik data spatial dan pengurusan datanya. Memori untuk menyimpan jumlah data yang besar dan mempunyai kualitas yang baik dengan resolusi tinggi pada skrin grafik warna (untuk membantu dalam menentukan maklumat yang dihasilkan atau diberikan melalui penggunaan warna yang berbeda).
    \item Metode GIS, merupakan prosedur dari analisis sistem GIS yang melibatkan proses input, proses menyimpan, proses mengurus, proses menukar, proses menganalisis, dan proses output yang hanya melibatkan per isian GIS untuk mengatur sistem dan data-data tersebut (Heywood et al., 2002 )
    \item People, merupakan orang-orang yang menggunakan sistem GIS, atau orang yang mengendalikan proses input-output sistem GIS.
\end{itemize} 


\subsection{Kaidah GIS atau Geography Information System}
Data spatial, pengurusan data atribut, paparan data, penerokaan data, analisis dan pemodelan data GIS;
Yang dijelaskan oleh gambar sebagai berikut:

\begin{figure}[ht]
	\centerline{\includegraphics[width=1\textwidth]{figures/kaedahGIS.JPG}}
	\caption{kaidah GIS}
	\label{kaidah GIS}
	\end{figure}


\begin{enumerate}
\item Input data spatial
Merupakan langkah awal agar terciptanya data baru, dengan cara menginputkan data dan sistem GIS akan menyuntingnya dalam bentuk transformasi geometri yang nantinya akan menghasilkannya kedalam bentuk hard copy. (Chang, 2008) 
(Heywood et al., 2002).

\item Pengurusan data artibut
Merupakan langkah selanjutnya agar sumber peta dapat dipindahkan kepada peta digital yang dapat dibaca oleh GIS.
(Chang, 2008) (Worboy \& Duckham, 2003) (Heywood et al., 2002)

\item Pengumpulan data
Merupakan aktivitas untuk proses melakukan eksplorasi lebih jauh dalam meneliti ciri kesamaa dalam suatu graf peta yang berbeda. (Worboy \& Duckham, 2003).

\item Analisis data
Merupakan cara untuk memaparkan dan memanipulasi data yang didapat. Dengan menggunakan 2 jenis format, yaitu :
\begin{itemize}
\item data vektor : melibatkan beberapa kaidah seperti penimbalan / buffering, penindihan/overlay, pengukuran jarak, statik ruang, dan manipulasi peta.
\item data raster : menganalisis pengumpulan data tempatan, kaidah kejiranan, kaidah berzon, dan kaidah operasi global.

(Chang, 2008) (Worboy \& Duckham, 2003) (Heywood et al., 2002)
\end{itemize}

\item Paparan data dan output data
Dasarnya disediakan untuk tujuan pemaparan hasil dari analisis data yang fungsinya ditujukan untuk pengguna.

\item Aplikasi GIS
Digunakan untuk keperluan tertentu dan bersifat umum bagi masyarakat tergantung keperluan penggunanya. 
(Heywood et al., 2002).
\end{enumerate}


Pada gambar \ref{aplikasi GIS} dijelaskan bahwa aplikasi GIS sesuai keperluan penggunaan sebagai berikut.

\begin{figure}[ht]
	\centerline{\includegraphics[width=1\textwidth]{figures/aplikasiGIS.JPG}}
	\caption{aplikasi GIS}
	\label{aplikasi GIS}
	\end{figure}

Maka artikel:
	Dalam sebuah artikel dari hua yang menyebutkan bahwa GIS memiliki kaidah dan komponen, Information dan System \cite{hua2017sistem}.


\subsection{Kesimpulan GIS atau Geography Information System}
Kesimpulannya, GIS merupakan alat yang penting dalam perspektif komputer pada masa kini dikarenakan GIS
mempunyai kemampuan aplikasi dalam berbagai bidang, misalnya dalam proses perancangan bandar dan kartografi,
penilaian kesan alam sekitar dan pengurusan sumber asli. GIS juga memainkan peranan dalam perspektif perniagaan,
dimana alat ini sangat bermanfaat dalam pengiklanan dan pemasaran, jualan, dan logistik 
mampu digunakan untuk mencari dan meningkatkan perniagaan seperti tapak perniagaan yang strategi. Sebagai umum, pengguna GIS dapat dilibatkan dengan agensi-agensi penguat kuasaan undang-undang, strategi
perancangan, perhutanan, industri, pemberdayaan alam, perencanaan kota, profesional
telekomunikasi, kesehatan, pengangkutan, geografi, dan pembangunan pemasaran. 
Penjelasan ini menyediakan platform untuk memahami lebih lanjut tentang komponen, kaidah, dan aplikasi GIS, 
untuk mempelajari tentang alat GIS.
\subsection{Saran GIS atau Geography Information System}

GIS dapat diaplikasikan di dalam kehidupan sehari-hari untuk memenuhi kebutuhan dan dapat membantu kebutuhan setiap masyarakat menjadi lebih baik dan lebih bermanfaat. Karena dengan memanfaatkan kemajuan teknologi maka teknologi yang digunakan akan ikut turut serta terus berkembang untuk menyesuaikan pemenuhan kebutuhan setiap pengguna yaitu masyarakat. Demikian kesimpulan dan saran yang dapat disampaikan kurang lebihnya mohon maaf dan terimakasih.

\subsection{Tipe Data Geospasial}
Ada dua tipe data geospasial. Data vektor dan data raster. Model data vektor merupakan model data yang paling banyak digunakan, model ini berbasiskan pada titik dengan koordinat (x,y) untuk membangun objek spasialnya. Objek yang dibangun
dibagi menjadi tiga bagian, yaitu: titik, garis, dan area (polygon).
Keuntungan dari data vector, yaitu: ketepatan dalam merepresentasikan fitur titik, batasan dan
garis lurus. Data raster adalah data yang dihasilkan dari sistem pengindraan yang jauh. Pada data raster,
objek geografis di representasi kan sebagai struktur sel grid yang disebut pixel. Resolusi pada data
raster tergantung pada ukuran pixel-nya.
Maka, resolusi pixel menggambarkan ukuran sebenarnya dari permukaan bumi yang diwakili
oleh setiap pixel pada citra. Semakin tinggi resolusi nya, semakin kecil permukaan bumi yang
di representasi kan oleh suatu sel. Data raster cocok untuk me representasi kan batas-batas yang
berubah secara gradual, seperti jenis tanah, vegetasi, suhu tanah, dan kelembaban tanah.

%\subsection{Analisis Pola Titik dan Estimasi Kepadatan}
%Saat berhadapan dengan pola titik, mendefinisikan acara sebagai lokasi pengamatan dalam distribusi, dan poin seperti semua lokasi lainnya di area belajar. Berbagai tingkat pengamatan dan analisis diajukan. Yang sederhana visualisasi suatu distribusi acara di luar angkasa melalui peta titik dapat disediakan informasi awal mengenai struktur distribusinya, namun lebih halus analitis Instrumen dibutuhkan lebih dalam analisis mendalam, dan terutama untuk mengidentifikasi kelompok atau keteraturan dalam distribusi relatif terhadap model yang diasumsikan, biasanya yang lengkap keacakan spasial (CSR).
%Analisis kuadrat merupakan salah satu cara untuk memesan pola distribusi kejadian di dalam suatu wilayah R. Ini melibatkan pembagian wilayah penelitian ke dalam sub-daerah yang sama dan homogen sebagai kuadrat dan kemudian menghitung jumlah kejadian yang jatuh di setiap sub-wilayah (kuadrat) guna menyederhanakan distribusi spasial.Jumlah kejadian jadi atribut kuadrat. Saat itulah memungkinkan untuk mewakili distribusi spasial dengan cara homogen dan mudah daerah yang sebanding, karena paket GIS memungkinkan untuk memvisualisasikan fenomena ini melalui tematik warna representasi kuadrat.
%Analisis yang berbeda dapat dihitung dan Hasil yang diperoleh, dengan mengubah asal grid atau dimensi. Satu perbaikan untuk ini Keterbatasan melibatkan mempertimbangkan jumlah kejadian untuk setiap unit area dalam ponsel `Jendela' radius tetap berpusat pada sejumlah titik di wilayah R. Sebuah perkiraan intensitas di setiap titik grid disediakan. Itu menghasilkan perkiraan dari variasi intensitas yang lebih halus dari yang diperoleh dari grid kuadrat tetap sel berlapis. Metode ini disebut metode `naif' dari sekelompok prosedur disebut Kernel Density Estimation (KDE).

