% Tugas GIS ke 6
% M. Amran Hakim Siregar 1154106
% Yusri Rizal 1154072


\section{Raster Layer dan Vektor Layer}

\subsection{Raster Layer}
Dalam raster layer  setiap lokasi direpresentasikan sebagai suatu posisi sel. 
Sel ini diorganisasikan dalam bentuk kolom dan baris sel-sel dan biasa disebut sebagai grid. 
Dengan kata lain,  layer raster menampilkan, menempatkan, 
dan menyimpan data spasial dengan menggunakan struktur matriks atau piksel-piksel yang membentuk grid. 
Setiap piksel atau sel ini memiliki atribut tersendiri, termasuk koordinatnya yang unik.
Raster layer terdiri dari piksel (atau sel), dan setiap piksel memiliki nilai yang terkait. Menyederhanakan sedikit, foto digital adalah contoh dataset raster dimana setiap nilai piksel sesuai dengan warna tertentu. Dalam GIS, nilai piksel dapat mewakili elevasi di atas permukaan laut, atau konsentrasi kimiawi, atau curah hujan dan lain-lain. Intinya adalah bahwa semua data ini diwakili sebagai kotak sel (biasanya persegi). Perbedaan antara model elevasi digital (DEM) di GIS dan foto digital adalah bahwa DEM mencakup informasi tambahan yang menjelaskan di mana tepi gambar berada di dunia nyata, bersamaan dengan seberapa besar setiap sel berada di tanah. Ini berarti GIS Anda dapat memposisikan gambar raster Anda (DEM, hillshade, peta lereng dll) dengan benar relatif terhadap satu sama lain, dan ini memungkinkan Anda membangun peta Anda.


\subsection{Vektor Layer}
Data vektor adalah data yang diperoleh dalam bentuk koordinat titik yang menampilkan, 
menempatkan dan menyimpan data spasial dengan menggunakan titik, garis atau area (poligon). 
Terdapat tiga tipe bentuk data vektor (titik, garis, dan poligon) yang bisa digunakan untuk menampilkan informasi pada peta. 
Titik bisa digunakan sebagai lokasi sebuah tempat atau posisi tertentu dalam peta. 
Garis bisa digunakan untuk menunjukkan route suatu perjalanan atau menggambarkan batas suatu wilayah 
dan juga batas suatu kawasan hutan atau area tertentu. 
Poligon bisa digunakan untuk menggambarkan sebuah danau atau sebuah luasan areal yang kemudian dapat analisis luasan 
pada areal-areal tersebut.

Vektor Layer terdiri dari masing-masing titik, yang (untuk data 2D) disimpan sebagai pasangan koordinat (x, y). 
Poin dapat digabungkan dalam urutan tertentu untuk membuat garis, atau bergabung dalam cincin tertutup untuk 
membuat poligon, namun semua data vektor pada dasarnya terdiri dari daftar koordinat yang menentukan simpul, 
beserta peraturan untuk menentukan apakah dan bagaimana simpul tersebut digabungkan. .
Perhatikan bahwa sementara data raster terdiri dari sel array yang teratur, 
titik-titik pada dataset vektor tidak boleh secara teratur spasi.

\subsection{Class}
Class adalah prototype, atau blueprint, atau rancangan yang mendefinisikan variable dan method-methode pada seluruh objek tertentu. 
Class berfungsi untuk menampung isi dari program yang akan di jalankan, di dalamnya berisi atribut / type data dan method untuk menjalankan suatu program.
Class merupakan suatu blueprint atau cetakan untuk menciptakan suatu instant dari  object. 
class juga merupakan grup suatu object dengan kemiripan attributes/properties, behaviour dan relasi ke object lain. 
Contoh : Class Person, Vehicle, Tree, Fruit dan lain-lain.

\subsection{STYLE Objects}
Yang dimaksud dengan objek pada java adalah sekumpulan software yang terdiri dari variable dan method-method yang terkait. 
Objek juga merupakan benda nyata yang di buat berdasarkan rancangan yang di definisikan di dalam class

Object adalah instance dari class. Jika class secara umum mepresentasikan (template) sebuah object, 
sebuah instance adalah representasi nyata dari class itu sendiri. Contoh : Dari class Fruit kita dapat membuat object Mangga, 
Pisang, Apel, dan lain-lain.

\subsection{MapServer}
MapServer adalah applikasi Open Source yang memungkinkan sebuah data peta diakses melalui web. Teknologi ini pertama kali dikembangkan oleh Universitas Minesotta Amerika Serikat. Hadirnya MapServer menjadikan pekerjaan membuat Peta Digital menjadi lebih mudah dan interaktif. Interaktif peta disini diartikan bahwa pengguna dapat dengan mudah melihat dan mengubah tampilan peta seperti zoom, rotate, dan menampilkan informasi (seperti menampilkan info jalan) dan analisis( seperti menentukan rute perjalanan) pada permukaan geografi. Diagram berikut menggambarkan bagaimana user berinteraksi dengan peta interaktif berbasis MapServer.
