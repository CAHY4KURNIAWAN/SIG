\section{Raster Layer dan Vektor Layer}

\subsection{Raster Layer}
Dalam raster layer  setiap lokasi direpresentasikan sebagai suatu posisi sel. 
Sel ini diorganisasikan dalam bentuk kolom dan baris sel-sel dan biasa disebut sebagai grid. 
Dengan kata lain,  layer raster menampilkan, menempatkan, 
dan menyimpan data spasial dengan menggunakan struktur matriks atau piksel-piksel yang membentuk grid. 
Setiap piksel atau sel ini memiliki atribut tersendiri, termasuk koordinatnya yang unik.


\subsection{Vektor Layer}
Data vektor adalah data yang diperoleh dalam bentuk koordinat titik yang menampilkan, menempatkan dan menyimpan data spasial dengan menggunakan titik, garis atau area (poligon). Terdapat tiga tipe bentuk data vektor (titik, garis, dan poligon) yang bisa digunakan untuk menampilkan informasi pada peta. Titik bisa digunakan sebagai lokasi sebuah tempat atau posisi tertentu dalam peta. Garis bisa digunakan untuk menunjukkan route suatu perjalanan atau menggambarkan batas suatu wilayah dan juga batas suatu kawasan hutan atau area tertentu. Poligon bisa digunakan untuk menggambarkan sebuah danau atau sebuah luasan areal yang kemudia dapat analisis luasan pada areal-areal tersebut.

Vektor Layer terdiri dari masing-masing titik, yang (untuk data 2D) disimpan sebagai pasangan koordinat (x, y). 
Poin dapat digabungkan dalam urutan tertentu untuk membuat garis, atau bergabung dalam cincin tertutup untuk 
membuat poligon, namun semua data vektor pada dasarnya terdiri dari daftar koordinat yang menentukan simpul, 
beserta peraturan untuk menentukan apakah dan bagaimana simpul tersebut digabungkan. .
Perhatikan bahwa sementara data raster terdiri dari sel array yang teratur, 
titik-titik pada dataset vektor tidak boleh secara teratur spasi.
