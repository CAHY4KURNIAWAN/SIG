% Kelompok Diagram Kartesius
% Rahmi Nurdin (1154109)
% Mustari Muammar (1154108)
% Fadillah Firdaus (1154103)

\section{Pengertian Diagram Kartesius}

Diagram Kartesius adalah sistem koordinat yang terdiri dari dua sumbu yang berisi titik-titik sebagai simbol relasi.
Domain sebagai sumbu horizontal dan ko domain sebagai sumbu vertikal.
Pada koordinat kartesius daerah asal (domain) diletakkan pada sumbu X (sumbu mendatar) dan daerah kawan (ko domain) diletakkan pada sumbu Y (sumbu tegak).
Sedangkan daerah hasilnya merupakan titik (noktah) koordinat pada diagram kartesius. Dari relasi di atas, dapat ditunjukkan diagram kartesius nya seperti di bawah:

\begin{figure}[ht]
	\centerline{\includegraphics[width=.5\textwidth]{figures/rahmi1.PNG}}
	\caption{hubungan antar titik pada diagram kartesius.}
	\label{rahmi1}
	\end{figure}

Diagram Kartesius merupakan suatu bangunan atas empat bagian yang batasi oleh dua buah garis yang berpotongan tegak lurus pada titik-titik (X, Y). 
Dimana X merupakan rata-rata dari rata-rata skor tingkat pelaksanaan atau kepuasan konsumen dari sebuah faktor atribut 
dan Y adalah rata-rata skor tingkat kepentingan seluruh faktor atau atribut yang mempengaruhi kepuasan konsumen.

Seluruhnya ada K faktor. Rumus berikutnya yang digunakan ada di gambar \ref{rahmi2}. Dimana K adalah Banyaknya faktor atau atribut yang mempengaruhi kepuasan konsumen.

\begin{figure}[ht]
	\centerline{\includegraphics[width=.5\textwidth]{figures/rahmi2.PNG}}
	\caption{rumus mencari K faktor.}
	\label{rahmi2}
	\end{figure}


Dimana: K = Banyaknya faktor atau atribut yang mempengaruhi kepuasan konsumen 
Diagram Kartesius	

\begin{figure}[ht]
	\centerline{\includegraphics[width=.5\textwidth]{figures/rahmi3.PNG}}
	\caption{penentuan kuadran pada diagram kartesius.}
	\label{rahmi3}
\end{figure}


Dari Gambar \ref{rahmi3} terbadi menjadi beberapa kuadran Kartesius.
\begin{itemize}
\item Kuadran A :
Pada posisi ini, jika dilihat dari kepentingan konsumen, atribut-atibut produk berada pada tingkat tinggi, tetapi jika dilihat dari kepuasannya, 

konsumen merasakan tingkat yang rendah, sehingga konsumen menuntut adanya perbaikan atribut tersebut.
\item Kuadran B :
Pada posisi ini, jika dilihat dari kepentingan konsumen, atribut-atribut produk berada pada tingkat tinggi, dan dilihat dari kepuasannya, 
konsumen merasakan tingkat yang tinggi juga.
\item Kuadran C :
Pada posisi ini, jika dilihat dari kepentingan konsumen, atribut-atribut produk kurang dianggap penting, tetapi jika dilihat dari tingkat kepuasan konsumen cukup baik.

Namun, konsumen mengabaikan atribut-atribut yang terletak pada posisi ini.
Kuadran D
Pada posisi ini, jika dilihat dari kepentingan konsumen, atribut-atribut produk kurang dianggap penting, tetapi jika dilihat dari tingkat kepuasannya, konsumen merasa

sangat puas.
\end{itemize}


\section{Penghitungan Rumus Diagram Kartesius}
\subsection{menghitung rumus, mencari titik}

Kartesius digunakan untuk menentukan tiap titik dalam bidang dengan menggunakan dua bilangan yang biasa disebut koordinat x dan koordinat y.
\begin{figure}[ht]

	\centerline{\includegraphics[width=1\textwidth]{figures/rahmi9.PNG}}
	\caption{penentuan titik pada kuadran kartesius.}

	\label{rahmi9}
\end{figure}


Sebuah titik dalam Diagram Kartesius, mengandung dua buah informasi yakni sumbu (x, y) seperti tampak pada Gambar 1.2. 
Yaitu titik (2, 3) adalah titik dimana nilai x=2 dan y=3. Daerah ini dikenal dengan kuadran I, dimana nilai x dan y adalah positif.

\begin{figure}[ht]
	\centerline{\includegraphics[width=1\textwidth]{figures/rahmi10.PNG}}
	\caption{penentuan garis pada kuadran kartesius.}
	\label{rahmi10}
	\end{figure}

Dari dua buah titik diagram kartesius, bisa ditarik menjadi sebuah garis(gambar \ref{rahmi10}). Artinya pada sebuah garis memiliki titik awal


\section{Contoh Penerapan/Pemetaan Diagram Kartesius}
Tujuan digunakannya diagram kartesius adalah untuk melihat secara lebih terperinci mengenai atribut-atribut yang perlu untuk dilakukan perbaikan. 

Langkah-langkah sebelum memetakan data ke diagram kartesius ini, adalah terlebih dahulu dengan menentukan nilai rata-rata setiap atribut yaitu X dan Y, 
dimana nilai perhitungannya telah kita peroleh dari perhitungan yang dilakukan sebelumnya.
Adapun hasil pembagian setiap atribut pada setiap kuadran ditampilkan pada gambar 2

\begin{figure}[ht]
	\centerline{\includegraphics[width=1\textwidth]{figures/rahmi4.PNG}}
	\caption{Diagram Kartesius}
	\label{rahmi4}
	\end{figure}

Setelah dilakukan perhitungan menggunakan diagram kartesius didapat hasil atribut-atribut yang harus diperbaiki adalah atribut yang berada pada kuadran A.

Adapun atribut yang harus diperbaiki pada kuadran A pada gambar \ref{rahmi5}.


\begin{figure}[ht]
	\centerline{\includegraphics[width=.5\textwidth]{figures/rahmi5.PNG}}
	\caption{Hasil Perhitungan Diagram Kartesius pada Kuadran A}
	\label{rahmi5}
	\end{figure}


Untuk atribut-atribut yang harus dipertahankan oleh pihak perusahaan setelah dilakukannya perhitungan menggunakan diagram kartesius adalah atribut-atribut
yang berada pada kuadran B, karena pada atribut yang berada pada kuadran B dianggap pelanggan sudah dapat memenuhi apa yang mereka inginkan. 
Adapun atribut yang harus dipertahankan dapat dilihat pada gambar \ref{rahmi6}.



\begin{figure}[ht]
	\centerline{\includegraphics[width=1\textwidth]{figures/rahmi6.PNG}}
	\caption{Hasil Perhitungan Diagram Kartesius pada Kuadran B}
	\label{rahmi6}
	\end{figure}

Atribut yang memiliki penilaian yang rendah karena atribut-atribut ini kurang dianggap penting oleh pelanggan dan perusahaan juga tidak memberikan pelayanan atau perhatian khusus, 
atribut ini dianggap tidak memberikan dampak yang besar bagi perusahaan.
Adapun atribut-atribut yang berada pada kuadran C dapat dilihat pada gambar \ref{rahmi7}.

\begin{figure}[ht]
	\centerline{\includegraphics[width=.5\textwidth]{figures/rahmi7.PNG}}
	\caption{Hasil Perhitungan Diagram Kartesius pada kuadran C}
	\label{rahmi7}
	\end{figure}

Untuk atribut yang ada pada kuadran D adalah atribut yang tidak dianggap penting bagi pelanggan, namun pihak perusahaan memberikan pelayanan yang berlebihan 
sehingga atribut ini dianggap berlebihan.
Adapun atribut yang berada pada kuadran D dapat dilihat pada gambar \ref{rahmi8}.	
\begin{figure}[ht]
	\centerline{\includegraphics[width=1\textwidth]{figures/rahmi8.PNG}}
	\caption{Hasil Perhitungan Diagram Kartesius pada Kuadran D}
	\label{rahmi8}
	\end{figure}



Diagram Kartesius
Dari hasil perhitungan yang telah dilakukan sebelumnya, terdapat 17 atribut yang perlu dilakukan perbaikan (Action) dan terdapat 10 atribut yang perlu mendapat
perhatian untuk dipertahankan oleh pihak perusahaan (Hold). Diagram Kartesius Dari hasil pemetaan yang dilakukan pada diagram kartesius dapat terlihat beberapa
atribut yang perlu untuk dilakukannya perbaikan dan atribut-atribut perlu untuk dipertahankan oleh pihak perusahaan yang terbagi kedalam kuadran-kuadran (A, B, C
dan D) sesuai dengan tingkat kesesuaian antara tingkat kepentingan pelanggan dan kinerja perusahaan, yaitu dengan tingkat kesesuaian sebesar 58.374.
Adapun hasil pemetaannya adalah sebagai berikut:
Kuadran A

Kuadran A adalah wilayah yang berisikan atribut-atribut yang dianggap penting oleh pelanggan, namun dalam kenyataannya atribut-atribut ini masih belum sesuai
dengan yang diharapkan oleh pelanggan. Dalam hal ini perusahaan perlu melakukan perbaikan sebaik mungkin untuk meningkatkan kepuasan pelanggan terhadap
atribut yang termasuk kedalam kuadran A. Dari diagram kartesius yang dibuat, diketahui bahwa atribut yang termasuk dalam kuadran A yaitu atribut 1, 3, 5, 6, 7,
10, 12, 14, 16, 26, 27.
Adapun beberapa hal yang sebaiknya perlu dilakukan guna perbaikan atau penyesuaian terhadap beberapa hal yang menjadi prioritas diatas yang pertama antara lain
perlunya dilakukan penambahan alat pendingin ruangan untuk dapat menjaga suhu ruangan demi kenyamanan pelanggan, Penambahan ukuran meja kasir agar barang-barang belanjaan yang telah dipilih
tidak merepotkan pelanggan ataupun kasir. Selain itu juga perlu dilakukannya perbaikan ataupun pembersihan ruangan toilet dan pendukung lainnya seperti ketersediaan air
sehingga pelanggan yang menggunakan akan merasa lebih nyaman, penambahan jumlah keranjang belanjaan yang disediakan perusahaan, Lebih melengkapi jenis-jenis
produk yang ditawarkan dengan mempertimbangkan tempat penyimpanan serta waktu-waktu tertentu seperti hari-hari besar nasional dan lain sebagainya, Memberikan pengarahan kepada para
karyawan mengenai pentingnya berinisiatif dalam melayani pelanggan yang membutuhkan bantuan tanpa harus dimintai tolong terlebih dahulu oleh pelanggan.
Dapat juga dilakukan penambahan papan informasi berupa lokasi produk yang tersedia untuk dapat mengurangi frekuensi terjadi atau timbulnya pertanyaan dari para
pelanggan mengenai produk yang akan mereka beli, perbaikan ataupun penyesuaian secara berkala antara label-label harga yang tertera pada produk yang ditawarkan dengan perubahan-perubahan
harga yang terjadi, penataan tempat parkir yang dapat dilakukan dengan memberikan garis-garis pembatas kendaraan, ataupun dengan menambahkan tukang parkir untuk
dapat menanggulangi keamanan dan penataan tempat parkir kendaraan, penyusunan program-program promo secara berkala, seperti pemberian diskon dengan jumlah pembelian tertentu ataupun dengan
memberikan voucer belanja dengan nilai tertentu untuk dapat lebih menarik pelanggan, dan sebaiknya perusahaan memiliki atau beberapa jenis produk tertentu
yang diunggulkan dengan harga yang lebih murah dibandingkan dengan kompetitor lainnya sebagai penarik.
\subsection{Kuadran B}
Kuadran B adalah daerah yang memuat atribut-atribut yang dianggap penting oleh pelanggan, dan atribut-atribut tersebut dianggap telah sesuai dengan keinginan

pelanggan sehingga tingkat kepuasan pelanggan relatif lebih tinggi, sehingga perlu untuk dipertahankan oleh pihak perusahaan karena sudah bisa memberikan pelayanan
sesuai dengan keinginan pelanggan sehingga konsumen merasa puas. Adapun atribut yang termasuk kedalam kuadran ini adalah: 11, 13, 21, 24, 25.
Kuadran C
Kuadran C adalah Daerah yang berisikan atribut-atribut yang dianggap kurang penting oleh pelanggan dan pada kenyataannya kinerja pihak perusahaan pun dinilai kurang memuaskan. Tetapi tidak

menutup kemungkinan Kuadran C pada waktu yang akan datang menjadi perhatian yang penting oleh pelanggan, sehingga perusahaan juga harus mempertimbangkan
hal tersebut. Adapun atribut yang termasuk kedalam kuadran ini adalah: 2, 4, 9, 15, 17, 18, 19, 22.
\subsection{Kuadran D}
Kuadran D adalah wilayah yang memuat atribut-atribut yang dianggap kurang penting oleh pelanggan dan kinerja yang dilakukan oleh pihak perusahaan dirasakan
terlalu tinggi atau berlebihan, sehingga perusahaan tidak perlu melakukan perbaikan. Adapun atribut yang termasuk kedalam kuadran ini adalah: 8, 20, 23.

\section{Pengertian Bidang atau Diagram Kartesius}

Dalam mempelajari materi himpunan, fungsi, dan persamaan garis lurus kita akan mengenal yang namanya bidang atau diagram Kartesius. Apa itu bidang atau diagram Kartesius?

Diagram Kartesius adalah sistem koordinat yang digunakan untuk meletakan titik pada penggambaran objek berdasarkan pemasukan nilai pada sumbu x dan nilai pada sumbu y dimana titik pertemuan ini nilai dari sumbu x dan sumbu y titik koordinat dibentuk. Jadi, diagram Kartesius digunakan untuk menentukan tiap titik dalam bidang dengan menggunakan dua bilangan yang biasa disebut koordinat x dan koordinat y dari titik tersebut. Di mana x disebut absis dan y disebut ordinat.


Titik-titik pada koordinat kartesius merupakan pasangan titik pada sumbu-x dan sumbu-y (x, y). Perpotongan antara sumbu-x dan sumbu-y di titik 0 (nol) disebut pusat koordinat. Untuk bagian atas sumbu y bernilai positif, sedangkan pada bagian bawah sumbu y bernilai negatif. Begitu juga pada sebelah kanan sumbu x bernilai positif, sedangkan pada sebelah kiri sumbu x bernilai negatif. Untuk contohnya silahkan lihat gambar di bawah ini. 

\begin{figure}[ht]
	\centerline{\includegraphics[width=.5\textwidth]{figures/cau100.PNG}}
	\caption{penentuan garis/titik dalam diagram kartesius}
	\label{cau100}
	\end{figure}

Perhatikan diagram Kartesius pada gambar di atas. Warna ungu (violet) merupakan pusat koordinat yaitu titik (0,0) yang artinya sumbu x dan y bernilai nol. Untuk warna hijau, pada sumbu x bernilai 2 dan sumbu y bernilai 3 maka koordinat dalam bidang kartesius ditulis (2,3). Untuk warna merah, pada sumbu x bernilai  – 3 dan sumbu y bernilai 1 maka koordinat dalam bidang kartesius ditulis (– 3, 1). Sedangkan untuk warna biru, pada sumbu x bernilai  – 3 dan sumbu y bernilai 1 maka koordinat dalam bidang kartesius ditulis (–1.5 , –2.5).

Menurut Wikipedia, istilah Kartesius digunakan untuk mengenang ahli matematika sekaligus filsuf dari Perancis bernama Descartes. Beliau memiliki peranan yang sangat besar dalam menggabungkan aljabar dan geometri (Kartesius adalah latinisasi untuk Descartes). Hasil kerjanya sangat berpengaruh dalam perkembangan geometri analytic, kalkulus, dan kartografi.


