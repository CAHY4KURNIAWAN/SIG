3A
1. metro extract jenis jenis data dan isinya 
2. QGIS instalasi dan penggunaan label, warna dkk
3. Instalasi Mapserver dan contoh nya pemanggilan di browser
4. Mapproxy instalasi, config yaml, dan running

3B

3C
Membuat fungsi yang kemudian di pull request ke github.com/awangga/gede
1. Melihat boundari box
2. Melihat type shapefile(poligon, point atau line)
3. Mencari dengan input nama jalan
4. mencari dengan input nama tempat

Commit per orang sehari satu selama 6 hari.

3D
1. Cara membaca boundary box menggunakan pyshp
2. cara membaca field kolom menggunakan pyshp
3. cara membaca data kordinat point, line, poligon menggunakan pyshp
4. cara membaca jumlah row dan mencari nama lokasi dengan menggunakan pyshp



1. Menggunakan standar git, dan latex lihat di menu standar portal
2. commit 1 per hari minimal 50 kata selama 6 hari. (60)
3. hasil nilai akhir dikali presentase uniqeness
4. terdapat figure standar (10)
5. terdapat table atau equation standar(10)
5. Referensi standar (20)