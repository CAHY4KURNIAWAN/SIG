tugas
3A
1. sejarah, definisi, identation dan instalasi python
2. jenis2 variable python dan cara penggunaannya
3. perulangan di python dengan contohnya
4. fungsi dan kelas di python dengan contohnya


3B 
1. tentang OGC
2. data Raster
3. data Vektor
4. wmts

3C
1. point
2. line
3. poligon
4. shapefile

3D
1. openlayer
2. leafletjs
3. wms
4. qgis




a. Menggunakan Aturan : (50)

		1. file disimpan dalam format namatugas.tex (5)

		2. gambar disimpan dalam folder figures dengan namagambar (5)

		3. referensi wajib dari google scholar,scholar.google.com (5)

		4. Setiap referensi yang diambil, maka tuliskan ke dalam 
			file bernama reference.bib
		   yang berisi kumpulan bibTex dari referensi nomor 3 (5)

		5. Gunakan standar pengutipan yang baik dan benar (5)

		6. Gambar disebutkan di dalam artikel dengan format \ref{namagambar}
		   dan gambar diselipkan dengan menambahkan blok sintaks(5) :
			\begin{figure}[ht]
			\centerline{\includegraphics[width=1\textwidth]{figures/namagambar.JPG}}
			\caption{penjelasan keterangan gambar.}
			\label{namagambar}
			\end{figure}
			Contoh :
			Pada gambar \ref{namagambar} dijelaskan bahwa sistem operasi memiliki 
			3 versi.
	
		7. Referensi disebutkan dengan menyebutkan nama di dalam file bibtex No.4
		   contoh (5) :
			Jika Bibtex :
			@inproceedings{ganapathi2006windows,
			  title={Windows XP Kernel Crash Analysis.},
			  author={Ganapathi, Archana and Ganapathi, Viji and Patterson, David A},
			  booktitle={LISA},
			  volume={6},
			  pages={49--159},
			  year={2006}
			}
			Maka artikel :
			Dalam sebuah artikel dari Ganapathi yang menyebutkan bahwa komputasi 
			adalah keniscayan \cite{ganapathi2006windows}.
	
	
		8. Penyebutan subbab dan subsubbab diatur dengan cara (5)
			judul sub bab \section{nama sub bab}
			judul sub sub bab ditulis dengan \subsection{judul sub sub bab}
			judul sub sub sub bab ditulis dengan \subsubsection{Judul sub sub sub bab}
			contoh :
			\section{Sejarah Peta}
			Perkembangan peta dunia tidak luput dari para ahli geografi dan kartografi. Peta dunia yang populer pada saat ini merupkan kontribusi dari para 
			pembuat peta sebelumnya

			\subsection{Ptolemy's}
			Ptolemy's diduga membuat peta pada abad ke 2
	
		9. Satu kelompok minimal 1500 kata, waktu maksimal 6*24 jam (5)

		10. Harus di scan plagiat setiap kelompok dari portal kampus keren, minimal uniqe 80% (5)






b. di github satu kelompok membuat organization nama kelompoknya dan masukan semua anggota ke grup tersebut (10)
c. fork buku gis https://github.com/BukuInformatika/SIG ke orgz tersebut (10)
d. commit sehari(min 50 kata) per anggota kelompok selama 6 hari (30)




Format Tambahan :
1. untuk list nomor gunakan
	contoh :
	berikut nama anggota kelompok
\begin{enumerate}
	\item darso
	\item karyo
	\item doyok
\end{enumerate}

2. spesial karakter menggunakan tanda \ didepannya
	contoh :
	\&
	\"dalam kurung\"
	jika spesial karakter menjadi banyak atau satu baris gunakan verb
	contoh :
	\verb|%$'%&$&'%'%'%&'%|
	
3. untuk tabel gunakan table , contoh

\begin{table}[h]
\caption{Small Table}
\centering
\begin{tabular}{ccc}
\hline
one&two&three\\
\hline
C&D&E\\
\hline
\end{tabular}
\end{table}

4. untuk rumus gunakan tag equation
	contoh:
	Rumus bola:

	a) Luas permukaan
	 \begin{equation}
	     L = 4 \pi r^2 \,
	\end{equation}
	b) Volume
	 \begin{equation}
	     V = \frac{4}{3}\pi r^3
	\end{equation}